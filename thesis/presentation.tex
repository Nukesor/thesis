\documentclass{beamer}

\usepackage{lipsum}
\usepackage[utf8]{inputenc}
\usepackage[ngerman,english]{babel}
\usepackage{amsmath}
\usepackage{amsthm}
\usepackage{graphicx}
\usepackage{caption}
\usepackage{lmodern}
\usepackage{float}
\usepackage{sidecap}
\usepackage{pgfplots}
\usepackage{pgfplotstable}
\usepackage{tabularcalc}
\usepackage{todonotes}
\usepackage{hyperref}
\usepackage{minted}
\usepackage{siunitx}
\usepackage{subfig}
% \usepackage{tabularx}
% \usepackage{setspace}
\usepackage[customcolors]{hf-tikz}
% \usepackage{url}
% \usepackage{csquotes}
\usepackage{booktabs}
% \usepackage[T1]{fontenc}
% \usepackage[alldates=long]{biblatex}
\graphicspath{{images/}}
\pgfplotsset{compat=1.12}

\title[Der Title]
{Das eine ding}
\subtitle{\textit{Or:} Die kleine variante?}
\author[Beer]{Arne Beer \\ \footnotesize Matriculation number: 6489196}
\institute[University of Hamburg]{
    Department of Computer Science\\
    University of Hamburg
}
\subject{Computer Science}

\usetheme{Szeged}
\usecolortheme{seagull}

\begin{document}
\frame{\titlepage}
\begin{frame}
    \frametitle{Table of Contents}
    \footnotesize
    \tableofcontents
\end{frame}

\section{Introduction}

\subsection{Topic}
\begin{frame}
    \frametitle{Topic}
    \begin{block}{Main topic of the thesis}
    \end{block}
\end{frame}

\subsection{Motivation}
\begin{frame}
    \frametitle{Motivation}
    \begin{itemize}
        \item
    \end{itemize}
\end{frame}

\subsection{Leading Question and Goals}
\begin{frame}
    \frametitle{Leading Question and Goals}
    \begin{itemize}
        \item
    \end{itemize}
\end{frame}

\begin{frame}
    \frametitle{Fin}
    Thank you for your attention.
\end{frame}
\end{document}
