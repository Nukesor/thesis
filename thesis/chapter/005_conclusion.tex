\chapter{Conclusion and Outlook}

The study set out to determine how feasible and precise data mining attacks on simple git metadata.
All three performed attacks, lead to promising results and showed potential for actual usage in a professional environment.
However, it must be noted that in all attacks only a small amount of the available data was used.
Simply using the git commit timestamps allowed us to perform analyses like narrowing down the location of a contributor.
The possible applications for the remaining data, such as actual changes in code, references of contributors between repositories or commit messages, are extensive.

If one would even add additional data from Github, such as followers, stars or information from their issues system, the results could further become substantially more accurate.
Developers around the world provide metadata about themselves on a daily basis, probably without knowing how much they are actually exposing.
To prevent the unauthorized usage and abuse of this data, we need to create counter-measures or prevent exposing this data in the first place.

The \ac{eu} set an example by enforcing the \ac{gdpr}, which is a regulation that strictly rules the handling of any user data.
