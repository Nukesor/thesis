\chapter{Conclusion and Outlook}

The study set out to determine how feasible and precise data mining attacks on simple Git metadata could be.
All three performed attacks, lead to promising results and showed potential for malicious usage.

The miss-out analysis showed that it is possible to automatically detect holiday and sick-leave anomalies.
Additionally, it is capable of detecting other anomalies in the developer's work pattern.

Analyzing the Git commit timestamps to narrow down the geographic location of a user-led to a significant reduction of possible locations on the globe.
With a proper test group, it is also likely to prove, that the other detected timezones represent the travel history of the target.

The analysis of punch cards showed, that it is possible to detect developers working at regular five day office hours and to distinguish between working employees and leisure time developers.

However, it must be noted that in all attacks only a small amount of the available data was used.
Simply using the Git commit timestamps allowed us to perform analyses such as narrowing down the location of a contributor.
The possible applications for the remaining data, like actual changes in code, references of contributors between repositories or commit messages, are extensive.

If one would add additional data from Github, such as followers, stars or information from their issues system, the results could become even more accurate.
Developers around the world provide metadata about themselves on a daily basis, probably without knowing how much they are actually exposing.
To prevent the unauthorized usage and abuse of this data, we need to create countermeasures or prevent exposing this data in the first place.

Luckily the \ac{eu} set an example by enforcing the \ac{gdpr}, which is a regulation that strictly rules the handling of any user data.
But there are still many countries left in the world, that do not have such strict rules and that might need ways to protect their privacy from being invaded.

While \emph{Gitalizer} is a foundation for data aggregation and the conduct of rather simple analyses, there is a necessity for more detailed research with better sources for ground truth.
Additionally, more statistics about the mining process would be convenient for evaluating the research results, such as the ratio between starred and contributed repositories.
\emph{Gitalizer} is a quite complex program, but it is well documented and should allow other people to easily jump into using it.

Many of the attacks mentioned in Section~\ref{attack-goals} are not implemented, as they did not fit in the scope of this thesis.
Implementing those could be the topic of another bachelor thesis or for a subsequent master thesis.

Furthermore, it would be interesting to explore the possibilities of countermeasures such as obfuscating Git commit timestamps.
