\section{Geographic location}

First of all, it needs to be clarified, that parts of this attack only works under specific circumstances.
Git commit timestamps are created by taking the current local time of the underlying \ac{os}.
If one wants to show the travel path of a target, the target's \ac{os} needs to automatically adjust the \ac{utc} offset accordingly to the current geographic location of the device.

This feature is available for newer versions of popular \acp{os}, such as \emph{Windows}~\footnote{Ivan Jenic, `Your Time Zone Can Now Switch Automatically in Windows 10', windowsreport.com, https://windowsreport.com/time-zone-automatic-switch-windows-10 (accessed, 24.04.2018)}
and \emph{Mac Os}, but they are not enabled by default.
It is also available for Linux, for instance with the \emph{tzupdate} package~\footnote{`Set the system timezone based on IP geolocation', github.com, https://github.com/cdown/tzupdate (accessed, 24.04.2018)}, but it needs to be manually installed and activated.

\begin{figure}[H]
    \includegraphics[scale=0.40]{./graphs/analysis/timezone-user-distribution}
    \centering
    \caption{Distribution of users according to the amount of timezone switches.}\label{fig:timezone-distribution}
\end{figure}

Figure~\ref{fig:timezone-distribution} shows the amount of contributors in relation to the amount of detected timezone switches.
On about 70\% of considered contributors only a single timezone has been detected, looking at the last year.
These 70\% do either not commit if they travel, their \acp{os} do not synchronize the timezone accordingly to their location or they simply did not travel in the last two years.

\begin{figure}[H]
    \includegraphics[scale=0.10]{./graphs/analysis/author-home-location}
    \centering
    \caption{Home location analysis of the author.}\label{fig:author-home-location}
\end{figure}

In Figure~\ref{fig:author-home-location} the visualized home location analysis of the author can be seen.
Regions marked in dark green are regions in which the contributor is likely to live.
The light green region represents the timezone of the home location.
As you can see in Figure~\ref{fig:author-home-location} the country French Guiana is also marked as a possible home location.
This problem occurs due to the several conversions between country names and codes, which were necessary as stated in Section~\ref{timezone-implementation}.
This misassignment only happens during the visualization process of the results and thereby does not affect the results of the analysis.

To evaluate the overall precision of the geolocation results, the correctness of the determined home locations are checked.
Github allows users to specify a string for their current home location, which is collected during the aggregation process.
Unfortunately there are no conventions on how this string has to look like.
Initially I tried to pass these strings to OpenStreetMap, but this resulted in too many wrongly assigned locations.
The data provided by the users was obviously too arbitrary and full of mistakes for the OpenStreetMap \ac{api} to handle.

As a result I decided to manually choose a subset of locations by looking for distinct identifiers in the location strings.
For instance, every home location of a contributor, that contained \emph{Germany} or \emph{Deutschland} in their location string, should be in the  timezone \inlinecode{Europe/Berlin}, which switches between \ac{cet} and \ac{cest}.
I created 14 such rules and was thus able to validate the home location of about 4200 contributors.
The assignment of the contributors home location was correct in about 82\% of the considered contributors.

\begin{table}[]
    \centering
    \begin{tabular}{lllll}
        Strings query        & Considered & Expected timezone    & Correct & Mean subset size \\
        & & & & \\
        San Francisco        & 772        & US/Pacific           & 639     & 12.07  \\
        NYC, NY, New York    & 485        & America/New\_York    & 366     & 23.60  \\
        India                & 148        & Asia/Colombo         & 127     & 4.01   \\
        UK, United Kingdom   & 667        & Europe/London        & 510     & 13.14  \\
        France               & 490        & Europe/Paris         & 421     & 31.59  \\
        New Zealand          & 59         & Pacific/Auckland     & 52      & 4.76   \\
        Germany, Deutschland & 976        & Europe/Berlin        & 846     & 31.48  \\
        Poland               & 180        & Europe/Warsaw        & 154     & 31.56  \\
        Italy                & 130        & Europe/Rome          & 112     & 31.39  \\
        Tokyo                & 61         & Pacific/Palau        & 48      & 8.51   \\
        Spain                & 129        & Europe/Madrid        & 116     & 32.83  \\
        Los Angeles          & 86         & America/Los\_Angeles & 76      & 12.24  \\
        Adelaide             & 2          & Australia/Adelaide   & 2       & 4.00   \\
        Mexico               & 13         & Mexico/General       & 7       & 11.62
    \end{tabular}
    \caption{Results of the home location evaluation}\label{home-location-table}
\end{table}

It needs to be noted, that the accuracy of this result is quite certainly deteriorated by location strings, which contain ambiguous information.
On manual review of the location strings, there were, for instance, strings such as \inlinecode{I love NYC}, which belongs to a developer living in Germany.
The result might also be deteriorated by contributors, which moved to another country in the last year, as we cannot detect those for sure.

Nevertheless an accuracy of 76\% clearly shows, that it is possible to narrow down the location of a contributor to a timezone and even to a subset of countries, see the last column in Table~\ref{home-location-table} for reference, by simply looking at the their git commit timestamps.
