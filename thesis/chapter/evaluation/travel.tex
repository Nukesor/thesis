\section{Geolocation}

First of all, it needs to be clarified, that parts of this attack only works under specific circumstances.
Git commit timestamps identical to the current local time of the underlying \ac{os}.
If one wants to show the travel path of a target, the target's \ac{os} needs to automatically adjust the \ac{utc} accordingly to the current geolocation of the device.

This feature is available for the newer versions of big \acp{os}, such as Windows~\footnote{Ivan Jenic, `Your Time Zone Can Now Switch Automatically in Windows 10', windowsreport.com, https://windowsreport.com/time-zone-automatic-switch-windows-10 (accessed, 24.04.2018)}
and Mac, but it is not necessarily enabled by default.
It is also available for Linux, for instance with the \emph{tzupdate} package~\footnote{`Set the system timezone based on IP geolocation', github.com, https://github.com/cdown/tzupdate (accessed, 24.04.2018)}, but it needs to be installed and activated manually.

\begin{figure}[H]
    \includegraphics[scale=0.10]{./graphs/analysis/author-home-location}
    \centering
    \caption{Home location analysis of the author.}\label{fig:author-home-location}
\end{figure}

In Figure~\ref{fig:author-home-location} the visualized home location analysis of the author can be seen.
Regions marked in dark green are regions, in which the contributor is likely to live.
The light green region represents the timezone of the home location.
As you can see in Figure~\ref{fig:author-home-location} the country French Guiana is also marked as a possible home location.
This problem occurs due to the several conversions between country names and codes, which were necessary as stated~\ref{timezone-implementation}.
This misassignment only happens during the visualization of the results and thereby doesn't affect the results of the analysis.

To evaluate the overall precision of the geolocation results, the correctness of the determined home location is checked.
Github allows users to specify a string for their current home, which is collected during the aggregation process.
Unfortunately there are no conventions on how this string has to look like.
Initially I tried to pass these strings to OpenStreetMap, but this resulted in too many wrongly assigned locations.
The data provided by the users was obviously too arbitrary and full of mistakes for the OpenStreetMap \ac{api} to handle.

As a result I decided to manually choose a subset of locations by looking for distinct identifiers in the location strings.
For instance, every home location of contributor, which contained \emph{Germany} or \emph{Deutschland} in their location string, should be in the \ac{utc} +1 timezone and contain the specific timezone \inlinecode{Europe/Berlin}.
I created INSERTHERE such rules and was thus able to validate the home location of about NUMBERHERE contributors.
The assignment of the contributors home location was correct in about 92\% of the considered contributors.


