Git is a code version control system which is used by most programmers on a daily basis these days.
According to the Eclipse Community Survey about 42.9\% of professional software developers used git in 2014 with an upward tendency~\cite{article:git-popularity}.
It is deployed in many if not most commercial and private projects and generally valued by its users.
It allows quick jumps between different versions of a project's code base and to manage and merge code from different sources to one upstream.

Several million users send new commits to their Git repositories every day.
On Github alone, the currently biggest open source platform, there exist about 25 million active repositories, a total of 67 million repositories and about 24 million users~\cite{article:github-statistics}.

Some well known projects and organizations use Git, for example Linux, Microsoft, Ansible and Facebook~\cite{article:github-statistics}.
Every repository contains the complete contribution history of every contributing user.
Each commit contains the full directory structure, a link to a blob for every file, a timestamp, a commit message from the author and more additional metadata.

This raises the question how much information is hidden in the metadata of a Git repository and which attack vectors could be introduced by mining this information, regarding a contributer or the owner of the repository.

The newly gained knowledge could be utilized by employers to spy on their employees.
It could be used by an unknown attacker who aims to obtain sensitive information about a company and its employees trough their open-source projects.
It is even possible that a privat person wants to monitor another person that regularly contributes to open-source repositories.

As there have not been any papers published about this specific topic or at least no public paper and Git plays such a crucial role in todays information technology, I want to investigate and evaluate this potential threat.
