Git is a \ac{vcs} used by most programmers on a daily basis these days.
Its purpose is to help developers version and manage the code base of their projects.
According to the Eclipse Community Survey, about 42.9\% of professional software developers used Git in 2014 with an upward tendency~\cite{article:git-popularity}.
It is deployed in many, if not most, commercial and private projects and generally valued by its users.
On top of this, it allows collaborating with thousands of contributors on the same project whilst maintaining a clear version history.

Several million users send new commits to their Git repositories every day.
On Github alone, the currently biggest open source platform, there exist about 25 million active repositories and a total of 67 million repositories~\cite{article:github-statistics}.

Some well-known projects and organizations use Git, for example, Linux, Microsoft, Ansible, and Facebook~\cite{article:github-statistics}.
Each of those repositories contains the complete contribution history of every contributing user and every contribution contains all changes, a timestamp, a message from the author and their email address.

This raises the question how much information is hidden in this metadata of a Git repository and which attack vectors could be introduced by analyzing this information.
Could it be used to harm or manipulate a contributor or maybe even a company?

The gained knowledge could be utilized by employers to spy on their employees.
It could be used by an unknown attacker, who aims to obtain sensitive information about a company and its employees through their open-source projects.
It is also imaginable, that a private individual uses this data to monitor another person, which regularly contributes to open-source repositories.

As there have not been any papers published about this specific topic yet or at least no public paper and as Git plays such a crucial role in today's software development, I want to investigate and evaluate this potential threat.
Furthermore, I want to create a foundation for future research and provide a first example of how such attacks might look like.

\section{Contents}
This thesis examines the process and capabilities of building a data mining software based on Git metadata.
After the introduction explaining the motivation, the approach and the goals of this thesis.
In Chapter 2 three different attack models will be constructed to determine possible attack goals.
In this context related work will be mentioned with respect to the attack goals.
Chapter 3 will explain the requirements to the data, existing solutions and the process of building an own aggregator, which collects data from Github.
Chapter 4 shows the actual the approach and algorithmic implementation of three different attacks.
Chapter 5 evaluates the results of the previously presented algorithms by comparing it with real-world ground truth and conducting small surveys.
In Chapter 6 the overall results of this research will then be discussed and an outlook will be provided.

\section{Motivation}
Each year more and more data is collected by employers.
A study conducted by \ac{mit} students shows, that data-driven companies, that collect data about everything in their company, are about 5\% more productive~\cite{article:management-revolution}, but this should never justify any invasion into their employees' privacy.
Privacy violation already goes as far as tracking medical records of employees, which is a service provided by the company \emph{Castlight}~\cite{article:medical-data}.
Some parties warn against surveillance of employees by the management and demand stricter handling of employee data~\cite{article:vermessung-belegschaft}

Generally, it can be said, that we need to take better care of our personal information and that people need to know in which ways they can leak information.
This thesis aims to show this on the example of the \ac{vcs} Git, which is by itself a very handy tool for code versioning of a project.
It was not developed with any malicious intent, but it might be used for such.

\section{Leading Questions and Goals}

The primary research objective of this thesis is to find out if data mining on Git metadata would be feasible.
This question is explored by looking at various data mining techniques and applying them to aggregated data from Github.

To gather data and to perform analysis on the data for the purpose of this thesis, a program called \emph{Gitalizer} is developed.
It is capable of continuously collecting data from Github in several ways.
Furthermore, three different analysis methodologies and several evaluation methods for these are implemented.
