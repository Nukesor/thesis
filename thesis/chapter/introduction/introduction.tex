Git is a \ac{VCS} used by most programmers on a daily basis these days.
It is supposed help developers to manage their
According to the Eclipse Community Survey about 42.9\% of professional software developers used git in 2014 with an upward tendency~\cite{article:git-popularity}.
It is deployed in many if not most commercial and private projects and generally valued by its users.
It allows to collaborate with thousands of contributors on the same project whilst maintaining order.

Several million users send new commits to their Git repositories every day.
On Github alone, the currently biggest open source platform, there exist about 25 million active repositories, a total of 67 million repositories and about 24 million users~\cite{article:github-statistics}.

Some well known projects and organizations use Git, for example Linux, Microsoft, Ansible and Facebook~\cite{article:github-statistics}.
But every of those repository contains the complete contribution history of every contributing user.
Each of those contributions contains all changes, a timestamp, a message from the author and their email address.

This raises the question how much information is hidden in the metadata of a Git repository and which attack vectors could be introduced by analysing this information.
Could this information be used to harm or manipulate a contributor or maybe even a company.

The newly gained knowledge could be utilized by employers to spy on their employees.
It could be used by an unknown attacker who aims to obtain sensitive information about a company and its employees through their open-source projects.
It is even possible that a privat individual wants to monitor another person, which regularly contributes to open-source repositories.

As there have not been any papers published about this specific topic yet or at least no public paper and as Git plays such a crucial role in todays information technology, I want to investigate and evaluate this potential threat.
Furthermore I want to create a foundation for future research and provide a first example on how such attacks might look like.
