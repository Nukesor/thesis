\chapter{Data Aggregation}\label{data-aggregation}

The biggest initial task for this thesis was the acquisition of data.
The data should be as extensive as possible, feature a high conjunction between contributers over several repositories to verify a possible connection between those and have realistic meta data.
Two different solutions came up for these requirements.

The first approach was to design an algorithm for automatic local generation of repositories.
The main problem with this solution is, that the visualization and data mining code might be highly optimized for this specific generation algorithm.
Real world data is noisy and inconsistent. Thereby the developed solution might have worked on the generated data, but would have probably failed on real world data.

The second solution was to get real world data from somewhere. The most obvious choice was to mine data from open source projects.
I chose Github for this purpose, as it hosts one of the biggest collection of open source projects and provides a great \ac{api} for querying Github's meta data.
A problem with this approach is that we don't have access to all important meta data, as for example the full list of members for organizations or the internal team structure of organizations.
Another problem are old email addresses, which are not related to any account anymore.
Even though some ground truth is missing, I decided to use this approach as it was still the most viable and promising way to gather as much ground truth and real world noise as possible.


\section{Structure of the Data}

Before we get to the data aggregator, I want to briefly explain the internal Git storage data structure and mechanisms, which are important for the purpose of this thesis~\cite{book:pro-git}.

Git, as most programmers know it, is a collection of high level abstraction tools to work with it's underlying \ac{fs}.
The most basic structure in Git is an \emph{blob} object.
A \emph{blob} object is a file, which has been added to a Git \ac{fs}. It is compressed and and saved in the \inlinecode{.git/objects} directory under the respective \ac{sha1} hash of the uncompressed file.
The probability of a \ac{sha1} collision is really low, roughly $10^{-45}$, even though Google managed to force a collision in an controlled environment in 2017~\footnote{Announcing the first SHA1 collision:~\url{https://security.googleblog.com/2017/02/announcing-first-sha1-collision.html} Retrieved Dec. 16, 2017}.

To represent a UNIX \ac{fs} or to simply bundle multiple Git \emph{blob} objects together, Git introduces the \emph{tree} object.
A \emph{tree} object is a file, which has a \ac{sha1} hash reference to all underlying \emph{blob} and \emph{tree} objects as well as their names and file permissions.
If a \emph{tree} holds a reference to another \emph{tree} it could be interpreted as a subdirectory.

\begin{minted}[linenos]{text}
    100644 blob 11d1ee77f9a23ffcb4afa860dd4b59187a9104e9	.gitignore
    040000 tree ac0f5960d9c5f662f18697029eca67fcea09a58c	expose
    100644 blob 61b5b2808cc2c8ab21bb9caa7d469e08f875277a	install.sh
    040000 tree 8aaf336db307bdcab2f082bd710b31ddb5f9ebd4	thesis
\end{minted}
\begingroup
\captionof{listing}{\emph{tree} file example\label{lst:raw-commit}.}
\endgroup

Now we come to the probably most important Git feature for this thesis; the \emph{commit}.
The commit is utilized to provide an exact representation of a state of the repository's files and directories.

\begin{minted}[linenos]{text}
    tree cd7d001b696db430b898b75c633686067e6f0b76
    parent c19b969705e5eae0ccca2cde1d8a98be1a1eab4d
    author Arne Beer <arne@twobeer.de> 1513434723 +0100
    committer Arne Beer <arne@twobeer.de> 1513434723 +0100
\end{minted}
\begingroup
\captionof{listing}{\emph{commit} file example\label{lst:raw-commit}.}
\endgroup

As you can see in listing~\ref{lst:raw-commit}, the \emph{commit} is just another kind of file utilized by Git, which contains some meta data about a repository version:

\begin{itemize}
    \item The reference to a \emph{tree} object. This is practically the root directory of the Git project
    \item A reference to one or multiple parent commits, to maintain a version history
    \item The name and email address of the author
    \item The name and email address of the committer
    \item The exact commit and publish \ac{utc} timestamp with timezone
\end{itemize}

Just as the \emph{blob} object the \emph{tree} and \emph{commit} files are also stored in the \inlinecode{.git/objects} directory under their respective hash.

With these simple methods Git manages to create a robust \ac{vcs}.
Git also provides tools to easily switch between commits of a project (checkout), show the changes between two different commits (diff) and to resolve conflicts between two different commits and merge them together.
There are a lot more features available, but those mentioned are the most important for this project.

\section{Github Meta Data}\label{github-meta-data}

I decided to use Github as a data source, because it is not only convenient to find \acp{url} for cloning repositories, but also provides some other useful meta data, which can be used to evaluate the precision of extracted knowledge.

Github offers some features, which are convenient to find repositories a specific user contributed to and to find other contributer which are likely related to each other.

The first feature is \emph{starring}. Every user can \emph{star} a repository to show that he likes a project.
The Github \emph{api} doesn't provide a method to get all repositories a user ever contributed to, it only allows to query the repositories owned by a user and the repositories \emph{starred} by a user.
With this feature it is possible to get some repositories a user contributed to, even though he doesn't own these repositories, as users tend to star repositories they contributed to (quote).

Another feature is \emph{following}. Every user can \emph{follow} another user to get informed, if they do specific things like creating new repositories or \emph{starring} repositories.
As user tend to (quote) \emph{follow} friends or colleagues, repositories of people, which are somehow personally related or work together, can be located.

The third feature are \emph{organizations}.
An organization is used to host projects under an account which is not necessarily lead by a single natural person, but rather supports roles with different permissions and team structures.
This feature provides us with some important ground truth, but sadly a lot of information is not visible, as users have to actively opt-in, if they want to be publicly displayed as a member of an organization.
Additionally team structures can only be examined, if one is a member of the organization.
Despite not knowing all members of an organization, we still get some useful information to estimate the tendency of precision of our knowledge extraction algorithms.


\section{Data structure}


\section{The Aggregator}

As mentioned in~\ref{data-aggregation}, I decided to get data from Github and wanted to utilize their \emph{Github APIv3} for this purpose.
This \ac{api} is publicly available and can be used by anyone who is registered on Github.
There is a rate limit of 5000 requests per user per hour.

The program I wrote for this thesis is named Gitalizer and features data aggregation, preprocessing, knowledge extraction and visualization.
For interaction with the Github \ac{api} the \emph{pygithub} library is used, which provides a convenient abstraction layer for requests and automatically maps \ac{json} responses to python objects.
The data aggregation module of Gitalizer is capable of several scanning methods.

\subsubsection{Stand-alone repository}\label{stand-alone-repository}
Gitalizer can scan any git repository from a \ac{ssh} or \ac{http} \acs{url} as long as it has access to it.
At first the repository is cloned into a local directory and when the cloning is don the scan process begins.
During the scan, we checkout the \emph{HEAD} of the current default branch for this repository and walk down every commit of the Git history.
The program saves all available meta data for each commit.
The emails, timestamps and names of the committer as well as additions and deletions to the project in lines of code.

After this scan we are still missing a lot of information.
The unique identifier of an author or committer is their email address, as names may change or can be ambiguous.
The problem with the simplicity of Git is that there exists no concept of a user.
Thereby we cannot easily link email addresses to a specific contributer.

\subsubsection{Github repository}
To tackle the problem in~\ref{stand-alone-repository}, I used the Github \ac{api} to get some of the missing meta data.
The general approach is the same as in the previous scan method. The repository is cloned and locally scanned.
However, a request to Github is issued every time a new email is found that we do not already have linked to a contributer.
Github allows to link multiple email addresses with a single user account and automatically references the respective user in their own \ac{api} commit representation.
With this additional meta data we gain ground truth about the identity of an author or committer.
Anyway this approach does not work, if the user of a commit removed the email used for the commit from his account, or if the user deleted his account.


\subsubsection{Github user}




\section{Problems}

