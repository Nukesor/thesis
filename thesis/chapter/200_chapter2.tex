\chapter{Attack models and their data requirements}\label{data-aggregation}

This chapter introduces three attacker models and their respective goals.
The required data to achieve and evaluate the goal will be listed and explained in the process.


\section{The employer}

This attack model deals with the scenario of an employer, which wants to monitor their employees.
The attacker's motivation is to spot irregularities in working behavior as well as unmotivated or unproductive employees.


\begin{description}
    \item[Productivity of Employees] \hfill \\
        Ensure employees produces enough code.
        For this purpose the changes in lines of code over a specific time span will be evaluated.
        \begin{itemlist}{Required data:}
            \item Commits of the employer's repositories.
            \item Commit timestamps
            \item Additions of each commit
            \item Deletions of each commit
        \end{itemlist}

    \item[Compliance of Working Hours] \hfill \\
        Check if an employee is productive in the defined working hours.
        This is especially useful to supervise employees, which work remotely.
        \begin{itemlist}{Required data:}
            \item Commits of employer's repositories.
            \item Commit timestamps
        \end{itemlist}

    \item[External Projects during Working Hours] \hfill
        Inspect if an employee is working on an external project during working hours.
        This only works if the employer has access to the external project, for example open source projects.
        \begin{itemlist}{Required data:}
            \item All commits of any available repository to come into question
            \item Commit timestamps
        \end{itemlist}

    \item[Code Quality Between Employees] \hfill
        Compare the quality of contributed code between different employees.
        With this metric the quality of an employee could be measured.
        To compare the quality we would need an external tool for code analysis.
        \begin{itemlist}{Required data:}
            \item Commits of the employer's repositories.
            \item Complete commit patch
            \item Commit timestamps
        \end{itemlist}
\end{description}


\section{The Industrial Spy}

This attack model covers the scenario of an external person, which wants to gain as much private or malicious information about a company as possible.
The attacker's motivation is either to harm the company, gain an advantage as an competitor or in the stock market or to sell secret information to a third party.
This attack vector only works if the targeted company is providing their product or at least parts of their product as open-source software.


\begin{description}
    \item[Company Employees] \hfill \\
        The most important target is to detect the company's employees as three other goals for this attacker model depend on this information.
        Another motivation could be to detect company members for further social engineering attacks or to headhunt the company's employees.
        \begin{itemlist}{Required data:}
            \item All commits of the company's repositories.
            \item Commit history graph.
        \end{itemlist}

    \item[Employee History] \hfill \\
        Check if an employee is productive in the defined working hours.
        This is especially useful to supervise employees, which work remotely.
        \begin{itemlist}{Required data:}
            \item Company Empllyees
        \end{itemlist}

    \item[Global Workforce Distribution] \hfill
        Inspect if an employee is working on an external project during working hours.
        This only works if the employer has access to the external project, for example open source projects.
        \begin{itemlist}{Required data:}
            \item All commits of any available repository to come into question
        \end{itemlist}

    \item[Status of the Product] \hfill
        Compare the quality of contributed code between different employees.
        With this metric the quality of an employee could be measured.
        To compare the quality we would need an external tool for code analysis.
        \begin{itemlist}{Required data:}
            \item Commits of the employer's repositories.
        \end{itemlist}

    \item[Internal team structures] \hfill
        Compare the quality of contributed code between different employees.
        With this metric the quality of an employee could be measured.
        To compare the quality we would need an external tool for code analysis.
        \begin{itemlist}{Required data:}
            \item Commits of the employer's repositories.
        \end{itemlist}
    \end{description}
