\chapter*{Abstract}

Software developers use many tools during their daily work and expose lots of data, often without being aware of doing so.
This data could be used to surveil, to spy on or to influence these developers.

Recent events as the Facebook scandal, in which the data of several million people has been exposed to a consulting company~\footnote{`Facebook scandal hits 87 million users' BBC.com, http://www.bbc.com/news/technology-43649018 (accessed, 24.04.2018)}, show how data can be abused to extract valuable knowledge and can be used for malicious purposes.

This thesis aims to give an example of how much information can be exposed by simply using the popular \ac{vcs} \emph{Git}.
Simple metadata such as UNIX timestamps and email addresses might be enough to extract sensitive information about users or organizations using Git.
This paper covers the whole process of gathering data from a vast amount of Git repositories, through to preprocessing, generating and interpreting the results of the analyses.
With this thesis, I hope to raise the awareness how dangerous it can be to expose even simple metadata and to proof that it can be used maliciously.
