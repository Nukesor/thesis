\chapter*{Abstract}


`As long as we use modern technologies we always expose data about ourselves'. This is a statement I truly believe in.

Recent events, such as the Facebook scandal in which the data of several million people has been exposed to a consulting company~\footnote{`Facebook scandal hit 87 million users' BBC.com, http://www.bbc.com/news/technology-43649018 (accessed, 24.04.2018)} show how large amounts of data can be abused to extract valuable knowledge and used for malicious purposes.

This thesis aims to give an example of how much information can be exposed by simply using the popular version control system Git.
Simple meta data such as UNIX timestamps and email addresses might be enough to extract sensitive information about Git users or organizations using Git.
This paper covers the whole process of gathering the data from a vast amount of git repositories through to preprocessing and interpreting the results of the analyses.
With this thesis I hope to raise the awareness how dangerous it can be to expose even simple meta data and that it might be used maliciously.
