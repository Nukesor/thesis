\chapter{Introduction}

Git is a code version control system which is used by most programmers on a daily basis these days. According to the Eclipse Community Survey about 42.9\% of professional software developers used git in 2014 with an upward tendency~\footnote{Ian Skerret. Exclipse Community Survey 2014 Results.\ \url{https://ianskerrett.wordpress.com/2014/06/23/eclipse-community-survey-2014-results/} Retrieved Oct. 25, 2017}.
It is deployed in many if not most commercial and private projects and generally valued by its users. It allows quick jumps between different versions of a project's code base and to manage and merge code from different sources to one upstream.

Several million users send new commits to their Git repositories every day.
On Github alone, the currently biggest open source platform, there exist about 25 million active repositories, a total of 67 million repositories and about 24 million users~\footnote{The State of the Octoverse 2017, Retrieved Oct. 25 2011, \url{https://octoverse.github.com/}}.

Some well known projects and organizations use Git, for example Linux\footnote{\url{https://github.com/torvalds/linux}, Retrieved Nov. 24 2017}, Google\footnote{\url{https://github.com/google}, Retrieved Nov. 24 2017}, Adobe\footnote{\url{https://github.com/adobe}, Retrieved Nov. 24 2017} and Paypal\footnote{\url{https://github.com/paypal}, Retrieved Nov. 24 2017}.
Every repository contains the complete contribution history of every contributing user.
Each commit contains the full directory structure, a link to a blob for every file, a timestamp, a commit message from the author and more additional metadata.

This raises the question how much information is hidden in the metadata of a Git repository and which attack vectors could be introduced by mining this information, regarding a contributer or the owner of the repository.

The newly gained knowledge could be utilized by employers to spy on their employees.
It could be used by an unknown attacker who aims to obtain sensitive information about a company and its employees trough their open-source projects.
It is even possible that a privat person wants to monitor another person that regularly contributes to open-source repositories.

As there have not been any papers published about this specific topic or at least no public paper and Git plays such a crucial role in todays information technology, I want to investigate and evaluate this potential threat.

\section{Motivation}

\section{Leading Questions and Goals}

The ambition of this thesis is to find possible attack vectors for knowledge extracted from git metadata of a single or multiple git repositories and analyze the possible damage potential.
For that purpose I will look at three realistic attacker models and try to get as much compromising and harmful knowledge for the objective of each specific model.

In the following three different attacker models with potential goals are listed.
Some goals will probably be extended, changed, added or removed during the research process.

\begin{description}
    \item[Employer]
        The employer tries to get as much information about its employees with the intention of spying on them:
        \begin{itemize}
            \item Direct comparison of productivity between employees
            \item Compliance of working hours
            \item Check if employees work on external projects during working hours
            \item Code quality between employees
        \end{itemize}

    \item[Industrial Espionage]
        The attacker tries to get as much information from the public open-source projects of a company:
        \begin{itemize}
            \item Company members
            \item History of all employees
            \item Overall status of a project
            \item A project's Code quality
            \item Internal team structures
        \end{itemize}

    \item[Individual]
        Somebody tries to get as much information about the personal life of a contributing individual:
        \begin{itemize}
            \item Sleeping rythm and daily routine
            \item Sick leave and holiday
            \item Interests
            \item Programming languages and skills
            \item Personal relationships between various programmers
        \end{itemize}

\end{description}

