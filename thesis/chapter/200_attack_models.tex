\chapter{Attack models and their data requirements}\label{attack-models}
This chapter introduces three attacker models and their respective goals.
The required data to achieve and evaluate the goal will be listed and explained in the process.


\section{The Employer}
This attack model deals with the scenario of an employer, which wants to monitor their employees.
The attacker's motivation is to spot irregularities in working behavior as well as unmotivated or unproductive employees.

\begin{description}
    \item[Productivity of Employees] \hfill \\
        Ensure employees produces enough code.
        For this purpose the changes in lines of code over a specific time span will be evaluated.
        \begin{itemlist}{Required data:}
            \item Commits of the employer's repositories.
            \item Commit timestamps
            \item Additions of each commit
            \item Deletions of each commit
        \end{itemlist}

    \item[Compliance of Working Hours] \hfill \\
        Check if an employee is productive in the defined working hours.
        This is especially useful to supervise employees, which work remotely.
        \begin{itemlist}{Required data:}
            \item Commits of employer's repositories.
            \item Commit timestamps
        \end{itemlist}

    \item[External Projects during Working Hours] \hfill
        Inspect if an employee is working on an external project during working hours.
        This only works if the employer has access to the external project, for example open source projects.
        \begin{itemlist}{Required data:}
            \item All commits of any available repository to come into question
            \item Commit timestamps
        \end{itemlist}

    \item[Code Quality Between Employees] \hfill
        Compare the quality of contributed code between different employees.
        With this metric the quality of an employee could be measured.
        To compare the quality we would need an external tool for code analysis.
        \begin{itemlist}{Required data:}
            \item Commits of the employer's repositories.
            \item Complete commit patch
            \item Commit timestamps
        \end{itemlist}
\end{description}



\section{The Individual}
This scenario describes a single person, which wants to harm, monitor or gain information about an open source developer.

An example goal of an attacker could be to either stalk the victim, harm him in any way or to manipulate him or one of his acquaintances.
The motivation of this attacker is mostly personal and on an emotional level.

Another non emotional attacker could be a robber trying to find the perfect time window to rob a house or the tracking of a high profile target.

A third attacker could be a headhunter which tries to get information about the skills and reliability of a developer.

\begin{description}
    \item[Sleeping Rhythm and Daily Routine] \hfill
        Learn about the persons sleep rhythm and obvious patterns in his daily routine.
        This attack aims to understand and predict the victim's behaviour.
        \begin{itemlist}{Required data:}
            \item Victim's commits.
            \item Commit timestamps.
        \end{itemlist}

    \item[Personal Relationships to Various Programmers] \hfill
        Predict possible friends and the likely grade of strength of their personal relationship.
        This could be crucial information for further social engineering attacks or to find similar skilled developer for headhunting.
        \begin{itemlist}{Required data:}
            \item Victim's commits.
            \item Commit timestamps.
        \end{itemlist}

    \item[Sick Leave and Holiday] \hfill
        Detect breaks in his typical work behaviour, which could represent holiday breaks or sick leave.
        This attack could give information about whether a developer is at home right now or if he tends to be sick alot.
        \begin{itemlist}{Required data:}
            \item Victim's commits.
            \item Commit timestamps.
        \end{itemlist}
\end{description}



\section{The Industrial Spy}
This attack model covers the scenario of an external person, which wants to gain as much private or malicious information about a company as possible.
The attacker's motivation is either to harm the company, gain an advantage as an competitor or in the stock market or to sell secret information to a third party.
This attack vector only works if the targeted company is providing their product or at least parts of their product as open-source software.

\begin{description}
    \item[Company Employees] \hfill \\
        The most important target is to detect the company's employees as three other goals for this attacker model depend on this information.
        Another motivation could be to detect company members for further social engineering attacks or to headhunt the company's employees.
        \begin{itemlist}{Required data:}
            \item All commits of the company's repositories.
            \item Commit history graph.
            \item As much meta data about the company's employees as possible for evaluation.
        \end{itemlist}

    \item[Employee History] \hfill \\
        Detect the timespan for which an employee worked at a given company.
        This could be interesting, as it shows the average employment duration and the employee amount over the history of the company, which could be an indicator of its current financial growth.
        Social engineering or headhunting could be a motivation here as well.
        \begin{itemlist}{Required data:}
            \item Company Employees
            \item Commit timestamps of the company's repositories
        \end{itemlist}

    \item[Global Workforce Distribution] \hfill
        Detect the timezone of all employees and create a global distribution graphic by timezones.
        This graphic allows you to guess the location of a company's workforce.
        It is also possible to create this statistic for all contributer, which could show a trend which countries or at least continents are interested the most for the company's product.
        \begin{itemlist}{Required data:}
            \item Company Employees
            \item All Commits
            \item Commit timestamps of the company's repositories
        \end{itemlist}

    \item[Internal Team Structures] \hfill
        Try to predict different teams, the role of each team and the respective team members.
        \begin{itemlist}{Required data:}
            \item Company Employees.
            \item Commits of the employer's repositories.
            \item Commit history graph.
        \end{itemlist}

    \item[Status of the Product] \hfill
        \todo{Überlegen, ob das mit reinkommt. Ist wahrscheinlich ein bisschen weit gefasst.}
        Compare the quality of contributed code between different employees.
        With this metric the quality of an employee could be measured.
        To compare the quality we would need an external tool for code analysis.
        \begin{itemlist}{Required data:}
            \item Commits of the employer's repositories.
        \end{itemlist}
\end{description}
