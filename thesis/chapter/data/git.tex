\section{Git}\label{git-explanation}
This section presents the \ac{vcs} \emph{Git}, as it plays a fundamental role in this thesis.
In the following, the most relevant parts of Git will be explained such as user roles, technologies, and internal data representations.
Moreover, current cases of application and some scenarios, that might be interesting for this thesis, will be mentioned.


\subsection{Introduction to Git}\label{git-introduction}
At its core, Git is a tool that is used to manage different versions of files in a specific directory.
A directory managed by Git is called a \emph{repository}.
Each version of the project is saved as a so-called \emph{commit}, which represents a specific state of all files and directories in the project.
Users are able to meticulously specify files or changes in files that should be added to a commit.
For instance, a developer can only commit a subset of the changes, which were applied to a repository.
By doing so, one can split a large set of possibly completely unrelated changes into several commits, where each commit in itself forms a set of logically related changes.
After creating at least two commits, Git is capable of showing the exact changes between any two commits, which is called a \emph{diff} and it allows to jump between different commits of the project, which is called a \emph{checkout}.

Git is the currently most popular tool to control a project's code with a still upward tendency~\cite{article:git-popularity}.
It enables to work with multiple developers on a single code base, as it provides several techniques to prevent, detect and resolve conflicts of changes at the same lines of code, namely the \emph{history tree}, the \emph{branch} and the \emph{merge}.
The commit history of Git is internally represented as a directed, non-cyclic, connected graph of commits.
The commits act as \emph{nodes} and the connection to their parent commits as \emph{edges}.

If a single node has two children, a new \emph{branch} has been created.
In Git a branch is a pointer to a specific commit, which allows you to easily jump between and to distinguish versions.
Git provides the feature to name branches, whereas the main branch is per default named \emph{master}.

\begin{figure}[H]
    \includegraphics[scale=0.35]{./graphs/git-history-branch}
    \centering
    \caption{A Git commit history tree that shows a history graph with two feature branches and a bug fix that is merged into the master branch.}\label{fig:git-commit-tree}
\end{figure}

As shown in figure~\ref{fig:git-commit-tree} two developers can, for example, create their own branch on which they can work unimpeded.
If they finished their tasks and want to add their work to the master branch, they can now merge their changes.
Git then tries to automatically resolve any conflicts which might have emerged from editing the same lines in a file.
If that is not possible, it marks the conflicts and allows the user to manually correct them.
After this resolution a new \emph{merge commit} is created.
A merge commit represents the merge of changes from two different branches.

With this methodology, it is possible to work with many people or teams on the same project, without accidentally overwriting changes of another developer, whilst maintaining a clear history of all changes in the project.

Another important feature of Git is the possibility to set up a \emph{remote}.
A remote often acts as a centralized repository, developers can \emph{push} their changes to or \emph{pull} changes from other developers.
This principle is called a \ac{cvcs} and is supported by many \acp{vcs}~\cite{version-control}.
It can, for example, be a distinct server, which is attached to some kind of network accessible by the developers.
This feature allows developers to remotely contribute to a project, as long as they have access to the remote's network.
Git also supports several protocols such as \ac{http} or \ac{ssh} to connect to a remote and thereby provide a simple user management layer.


\subsection{Git User Roles}
There exist two roles in Git, the \emph{committer} and the \emph{author}.
Every commit in Git contains the email addresses and the names of those two roles.
The author of a commit is the person who actually contributed the changes in the files.
The committer is the person, who created the Git commit.
This is important to keep track of the original author of the changes.

If one looks at the case of an author contributing code to a project via an email with an attached patch file.
If a maintainer of the project now applies the patch file and commits without setting the author, the information about the original author would be lost.
The collected data of 49 million commits, indicate that in about 89\% the author and the committer are the same person.


\subsection{Internal Representation}~\label{git-internal-representation}
Git's underlying storage and management solution for files is commonly described as a mini \ac{fs}~\cite[p.~9]{book:pro-git}
In the following I will thereby refer to its file management layer as \emph{git-fs} and explain the most important aspects of it.

The representation of a single file in Git is called a \emph{blob} object~\cite[p.~56]{book:pro-git}.
A blob object is a file, which has been added to a \emph{git-fs}.
It is compressed and saved in the \inlinecode{.git/objects} directory under the respective \ac{sha1} hash of the uncompressed file.
As a result, there exists a blob object for every version of every file of the project.

The \ac{sha1} hashing for unique file identifier might seem unsafe at first, but the probability of a \ac{sha1} collision is really low, roughly $10^{-45}$.
In 2017 Google managed to force a collision in a controlled environment, but it is really unlikely to encounter such a collision under normal circumstances~\cite{techreport:sha-collision}.
This characteristic of \ac{sha1} hashing will become quite important in the design of the database later on.

As mentioned in the Section~\ref{git-introduction} Git is used to store the state of a specific directory of an actual \ac{fs}.
To represent a \ac{fs} or to simply bundle multiple Git blob objects together, Git uses the tree object.

A tree object is a file, which has a \ac{sha1} hash reference to all underlying blob and tree objects as well as their names and their \ac{fs} permissions.
To represent a subdirectory, a tree simply holds a reference to another tree object.
With these tools, Git manages to build its own basic representation of a \ac{fs}.

\begin{minted}[linenos]{text}
    100644 blob 11d1ee77f9a23ffcb4afa860dd4b59187a9104e9  .gitignore
    040000 tree ac0f5960d9c5f662f18697029eca67fcea09a58c  expose
    100644 blob 61b5b2808cc2c8ab21bb9caa7d469e08f875277a  install.sh
    040000 tree 8aaf336db307bdcab2f082bd710b31ddb5f9ebd4  thesis
\end{minted}
\begingroup
\captionof{listing}{An example of a Git tree file\label{lst:raw-commit}.}
\endgroup

As stated before the commit is utilized to provide an exact representation of a state of the repository's files and directories.
For this purpose, it holds a reference to the tree object representing the root directory of the project as can be seen in line 1 in Listing~\ref{lst:raw-commit}.
Just as blob objects, the tree and commit files are also stored in the \inlinecode{.git/objects} directory under their respective hash.

\begin{minted}[linenos]{text}
    tree      cd7d001b696db430b898b75c633686067e6f0b76
    parent    c19b969705e5eae0ccca2cde1d8a98be1a1eab4d
    author    Arne Beer <arne@twobeer.de> 1513434723 +0100
    committer Arne Beer <arne@twobeer.de> 1513434723 +0100

    Chapter 2, acronyms
\end{minted}
\begingroup
\captionof{listing}{An example of a Git commit file\label{lst:raw-commit}.}
\endgroup

As you can see in Listing~\ref{lst:raw-commit}, the commit is just another kind of file utilized by Git, which contains some metadata about a repository version:

\begin{itemize}
    \item The reference to a tree object, which represents the root directory of the commit's version of the project.
    \item A reference to one or multiple parent commits, to maintain a version history.
    \item The name and email address of the author.
    \item The name and email address of the committer.
    \item The timestamps with an \ac{utc} offset for the committer and the author.
    \item The commit message. A message with arbitrary text from the committer.
\end{itemize}

The commit is the most important object for this thesis.
It contains crucial information such as the date of the commit as well as the email, which is needed to identify a contributor across several commits.

\subsection{Relevant Features}~\label{git-features}
Git provides a collection of high level abstraction tools to work with its underlying \emph{git-fs}.
In the following, the most important features for data aggregation will be listed and briefly explained.

\begin{description}
    \item[checkout] \hfill \\
        Git allows jumping between different project versions with the \emph{checkout} command.
        Calling \inlinecode{git checkout \$IDENTIFIER} jumps to the commit specified by the variable \inlinecode{\$IDENTIFIER}.
        The identifier, for instance, can be a partial or full \ac{sha1} hash of commit, a branch name or a tag.

    \item[HEAD] \hfill \\
        To mark the current checkout of a repository, Git utilizes a special marker called \emph{HEAD}.
        Due to this feature, it is, for instance, possible to jump to the previous commit in history with \inlinecode{git checkout HEAD\textasciitilde1}.

    \item[diff] \hfill \\
        The \emph{diff} is a tool to compare the state of two different commits in a repository.
        It allows listing all files which changed between those commits as well as showing the exact changes in the files

\end{description}

Git provides many more features, which are not necessarily important for data analysis, but which might need be taken into account when aggregating the data.
In the following, some functionalities will be shown, that can lead to problems or irregularities in the gathered data.

\begin{description}
    \item[force push] \hfill \\
        Git allows pushing to a remote with the \inlinecode{--force} flag, which is called a \emph{force push}.
        This enables the users to rewrite every commit in the whole history tree.
        If another user has the old Git repository version before the force push, they would now be incapable of simply pulling the newest version.
        Instead, they would need to manually checkout the newest commit of the rewritten remote branch and change their branch pointer to the new commit.

    \item[rebase] \hfill \\
        It is possible to \emph{rebase} branches. For instance, a rebase can rewrite the commit history and change the branch point of a branch to another commit.
        This is, for example, a very powerful but also dangerous tool, as it also implicitly changes the timestamps of the commits of the rebased branch.

    \item[filter-branch] \hfill \\
        In case somebody pushes a huge file, which significantly increases the size of the repository, or adds a file with secret information, such as a password file, Git provides the \emph{filter-branch} command.
        The filter-branch command removes a specified file from every commit in the history and thereby rewrites the history to the point, where this file was added to the repository.
        This command leads to similar problems as the rebase command since it can severely change the history of a repository.

\end{description}
