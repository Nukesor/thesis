\chapter{Git}\label{git-explanation}
In this chapter the \ac{vcs} \emph{Git} will be introduced. As Git and its techniques and data structures are the foundation of this thesis, I will explain the important user roles, technologies, and other interesting parts of Git.


\section{Introduction to Git}
Git is a tool, which is used to manage different versions of files in a specific directory.
Each version of the project is saved in a so called \emph{commit}.
Users are able to meticulously specify the files or changes of files that should be added to a commit. It is then capable of showing the exact changes between different commits, which is called a \emph{diff}.

Git is the currently \todo{Quote} most popular tool to control the code of a programming project.
It enables to work with multiple developers on a single code base, as it provides two different techniques, the \emph{branch} and the \emph{merge}.
The versioning of Git is internally represented as an non-cyclic, connected graph of commits or a \emph{tree}.
The commits act as \emph{nodes} and the connection to their parent commits as \emph{edges}.
Every time two edges leave a single node, a new \emph{branch} is created.
In Git, every branch has its own name, whereby the main branch is usually named \emph{master}.

In case two different people want to work on the same files, they can each create their own branch on which they can work unimpeded.
After they finished and want to add their work to the master branch, they can now \emph{merge} their changes.
Git then tries to automatically resolve any conflicts which might have emerged from editing the same lines in a file.
If that is not possible, it marks the conflicts and allows the user to manually correct them.


With this methodology it is possible to work with many people or teams on the same project without accidentally overwriting changes of another developer whilst maintaining a clear history of all changes of the project.

Another important feature of Git is the \emph{remote}.
A remote is usually located on a distinct server, which is attached to some kind of network, which is accessible by developers.
A remote acts as a single source of truth a developer can \emph{push} their changes to or \emph{pull} changes from other developers.
Git supports several protocols such as \ac{http} or \ac{ssh} to connect to the remote and to provide a simple user management layer.


\section{Git User Roles}
There exist two roles in Git, namely the \emph{committer} and the \emph{author}.
Every commit in Git contains the email addresses and the names of these two people.
The \emph{author} of a commit is the person which actually contributed the changes in the files.
The \emph{committer} is the person, which created the git commit.
This is important to keep track of the original author of the changes.
Lets look at the case of an author contributing code to a project in an email with an attached patch file.
If a maintainer of the project now applies the patch file and commits without setting the \emph{author}, the information about the original author would be lost.
Although in most cases the \emph{author} and the \emph{committer} are the same person. \todo{Prozentzahl auf Basis der gesammelten Daten.}
