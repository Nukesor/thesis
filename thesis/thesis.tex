\documentclass{thesis}

\usepackage{lipsum}
\usepackage[utf8]{inputenc}
\usepackage[ngerman,english]{babel}
\usepackage{amsmath}
\usepackage{amsthm}
\usepackage{graphicx}
\usepackage{caption}
\usepackage{lmodern}
\usepackage{float}
\usepackage{sidecap}
\usepackage{pgfplots}
\usepackage{pgfplotstable}
\usepackage{tabularcalc}
\usepackage{todonotes}
\usepackage{hyperref}
\usepackage{minted}
\usepackage{siunitx}
\usepackage{acronym}
\usepackage{subfig}
\usepackage{tabularx}
\usepackage{setspace}
\usepackage[customcolors]{hf-tikz}
\usepackage{url}
\usepackage{csquotes}
\usepackage{booktabs}
\usepackage[T1]{fontenc}
\usepackage[alldates=long]{biblatex}
\addbibresource{thesis.bib}
\graphicspath{ {images/} }
\pgfplotsset{compat=1.12}

\title{Privacy implications of exposing Git meta data}
\author{Arne Beer}

\pagenumbering{roman}
\begin{document}

\begin{titlepage}
    \includegraphics[width=6.8cm]{./pic/up-uhh-logo-u-2010-u-farbe-u-rgb.pdf}
    \begin{center}\Large
        % Universität Hamburg \par
        % Fachbereich Informatik
        \vfill
        Bachelor thesis
        \vfill

        \makeatletter
        {\Large\textsf{\textbf{\@title}}\par}
        \makeatother

        \vfill
        presented by
        \par\bigskip

        \makeatletter
        {\@author} \par
        \makeatother

        born on the 21st of December 1992 in Hadamar \par
        Matriculation number: 6489196 \par
        Department of Computer Science
        \vfill

        \makeatletter
        submitted on {\@date}
        \makeatother

        \vfill
        Supervisor: Dipl.-Inform. Christian Burkert \par
        Primary Referee: Prof.\ Dr.-Ing. Hannes Federrath \par
        Secondary Referee: Prof.\ Dr.\ Dominik Herrmann

    \end{center}
\end{titlepage}

\cleardoublepage{}

\chapter*{Abstract}

Software developers use many tools during their daily work and expose lots of data, often without being aware of doing so.
This data could be used to surveil, to spy on or to influence these developers.

Recent events as the Facebook scandal, in which the data of several million people has been exposed to a consulting company~\footnote{`Facebook scandal hits 87 million users' BBC.com, http://www.bbc.com/news/technology-43649018 (accessed, 24.04.2018)}, show how data can be abused to extract valuable knowledge and can be used for malicious purposes.

This thesis aims to give an example of how much information can be exposed by simply using the popular \ac{vcs} \emph{Git}.
Simple metadata such as UNIX timestamps and email addresses might be enough to extract sensitive information about users or organizations using Git.
This paper covers the whole process of gathering data from a vast amount of Git repositories, through to preprocessing, generating and interpreting the results of the analyses.
With this thesis, I hope to raise the awareness how dangerous it can be to expose even simple metadata and to proof that it can be used maliciously.

\clearpage

\vspace*{\fill}
\thispagestyle{empty}
\begin{quotation}
    \em
    Gotta go fast.

    \medskip
\raggedleft{}
    Senic the Herdgherd
\end{quotation}
\vspace*{\fill}


{\small \tableofcontents}

\chapter*{Acronyms}
\begin{acronym}
    \acro{api}[API]{Application programming interface}
    \acro{fs}[FS]{file system}
    \acro{sha1}[SHA-1]{Secure Hash Algorithm 1}
    \acro{url}[URL]{Uniform Resource Locator}
    \acroplural{url}[URLs]{Uniform Resource Locators}
    \acro{utc}[UTC]{Coordinated Universal Time}
    \acro{vcs}[VCS]{version control system}
\end{acronym}


\chapter{Introduction}
\pagenumbering{arabic}
Git is a \ac{VCS} used by most programmers on a daily basis these days.
It is supposed help developers to manage their
According to the Eclipse Community Survey about 42.9\% of professional software developers used git in 2014 with an upward tendency~\cite{article:git-popularity}.
It is deployed in many if not most commercial and private projects and generally valued by its users.
It allows to collaborate with thousands of contributors on the same project whilst maintaining order.

Several million users send new commits to their Git repositories every day.
On Github alone, the currently biggest open source platform, there exist about 25 million active repositories, a total of 67 million repositories and about 24 million users~\cite{article:github-statistics}.

Some well known projects and organizations use Git, for example Linux, Microsoft, Ansible and Facebook~\cite{article:github-statistics}.
But every of those repository contains the complete contribution history of every contributing user.
Each of those contributions contains all changes, a timestamp, a message from the author and their email address.

This raises the question how much information is hidden in the metadata of a Git repository and which attack vectors could be introduced by analysing this information.
Could this information be used to harm or manipulate a contributor or maybe even a company.

The newly gained knowledge could be utilized by employers to spy on their employees.
It could be used by an unknown attacker who aims to obtain sensitive information about a company and its employees through their open-source projects.
It is even possible that a privat individual wants to monitor another person, which regularly contributes to open-source repositories.

As there have not been any papers published about this specific topic yet or at least no public paper and as Git plays such a crucial role in todays information technology, I want to investigate and evaluate this potential threat.
Furthermore I want to create a foundation for future research and provide a first example on how such attacks might look like.


\section{Attack models}\label{attack-models}
To show possible scenarios, this section presents three attacker models and their respective goals.
These attack models serve as a guideline for the data aggregation process, which will be covered in the next chapter.
Only a small subset of these listed attacks will be actually performed by me, but this listing also serves as a exhibition of some possible attacks, for anyone that wants to further investigate this topic.

\subsection{The Employer}\label{employer-monitoring}
This attack model deals with the scenario of an employer, which wants to monitor their employees.
The attacker's motivation is to spot irregularities in working behavior as well as unmotivated or unproductive employees.

\begin{description}
    \item[Productivity of Employees] \hfill \\
        An employer wants to ensure that their employees produce sufficient amount of code.
        For this purpose the changes in lines of code over a specific time span will be evaluated.
        \begin{itemlist}{Required data:}
            \item Commit timestamps
            \item Additions of each commit
            \item Deletions of each commit
        \end{itemlist}

    \item[Compliance of Working Hours] \hfill \\
        Check if an employee is productive in the defined working hours.
        This is especially useful to supervise employees, which work remotely.
        \begin{itemlist}{Required data:}
            \item Commit timestamps
        \end{itemlist}

    \item[External Projects during Working Hours] \hfill \\
        Inspect if an employee is working on an external project during working hours.
        This only works if the employer has access to the external project, for example open source projects.
        \begin{itemlist}{Required data:}
            \item Commit timestamps
        \end{itemlist}

    \item[Code Quality Between Employees] \hfill \\
        Compare the quality of contributed code between different employees.
        With this metric the quality of an employee could be measured.
        To compare the quality we would need an external tool for code analysis.
        \begin{itemlist}{Required data:}
            \item Complete commit patch
            \item Commit timestamps
        \end{itemlist}
\end{description}



\subsection{The Individual}
This scenario describes a single person, which wants to harm, monitor or gain information about an open-source developer.

A possible goal of the attacker could be to either stalk the victim, harm them in any way or to manipulate them or one of his acquaintances.
The motivation of this attacker is mostly personal and on an emotional level.
Another non emotional attacker motive could be a robber trying to find the perfect time window to rob a house or the tracking of a high profile target.
A third attacker motive could be a headhunter which tries to get information about the skills and reliability of a developer.

\begin{description}
    \item[Sleeping Rhythm and Daily Routine] \hfill \\
        Learn about the persons sleep rhythm and obvious patterns in his daily routine.
        This attack aims to understand and predict the victim's behaviour.
        \begin{itemlist}{Required data:}
            \item Commit timestamps.
        \end{itemlist}

    \item[Personal Relationships to Various Programmers] \hfill \\
        Detect which contributor work together and try to discover the relationships between them.
        \begin{itemlist}{Required data:}
            \item Git history.
            \item Commit timestamps.
        \end{itemlist}

    \item[Sick Leave and Holiday] \hfill \\
        Detect breaks in his typical work behaviour. Those could represent holiday breaks or sick leave.
        This attack could give information about whether a developer is at home right now or if they tends to be sick a lot.
        \begin{itemlist}{Required data:}
            \item Commit timestamps.
        \end{itemlist}

    \item[Geolocation] \hfill \\
        Pinpoint the location of an contributor or at least narrow it down to a timezone or country could be interesting.
        Doing so could provide the attacker with a history of the target's travels.
        \begin{itemlist}{Required data:}
            \item Commit timestamps.
        \end{itemlist}
\end{description}



\subsection{The Industrial Spy}\label{industrial-spy}
This attack model covers the scenario of an external person, which wants to gain as much private or malicious information about a company as possible.
The attacker's motivation is either to harm the company, gain an advantage as an competitor or in the stock market or to sell secret information to a third party.
This attack vector only works if the targeted company is providing their full product or at least parts of their product as open-source software.

\begin{description}
    \item[Company Employees] \hfill \\
        The most important target is to detect the company's employees as three other goals for this attacker model depend on this information.
        Another motivation could be to detect company members for further social engineering attacks or to headhunt the company's employees.
        \begin{itemlist}{Required data:}
            \item Commit history graph.
        \end{itemlist}

    \item[Employee History] \hfill \\
        Detect the timespan for which an employee worked at a given company.
        This could be interesting, as it shows the average employment duration and the employee amount over the history of the company, which could be an indicator of its current financial growth.
        Social engineering or headhunting could be a motivation here as well.
        \begin{itemlist}{Required data:}
            \item Company Employees
            \item Commit timestamps
        \end{itemlist}

    \item[Global Workforce Distribution] \hfill \\
        Detect the timezone of all employees and create a global distribution graphic by timezones.
        This graphic allows you to guess the location of a company's workforce.
        It is also possible to create this statistic for all contributor, which could show a trend which countries or at least continents are interested the most for the company's product.
        \begin{itemlist}{Required data:}
            \item Company Employees
            \item All Commits
            \item Commit timestamps
        \end{itemlist}

    \item[Internal Team Structures] \hfill \\
        Try to predict different teams, the role of each team and the respective team members.
        \begin{itemlist}{Required data:}
            \item Company Employees.
            \item Commit history graph.
        \end{itemlist}
\end{description}


\chapter{Data}\label{data}
This chapter will attend to the collection of required data as stated in Section~\ref{attack-models}.
At first the \ac{vcs} \emph{Git} will be introduced and its functionalities explained.
The actual source of the data \emph{Github} will then be evaluated in terms of amount of ground truth and availability.
At last the methodology used for aggregation and exploration of Github will be explained.

\section{Git}\label{git-explanation}
This chapter introduces the \ac{vcs} \emph{Git}, as it plays a fundamental role in this thesis.
In the following the most relevant parts of Git will be explained such as user roles, technologies and internal data representations.
I will also talk about the current cases of application and some interesting scenarios which might be interesting for this thesis.


\subsection{Introduction to Git}\label{git-introduction}
At its core, Git is a tool, which is used to manage different versions of files in a specific directory. This
Each version of the project is saved as a so called \emph{commit}, which represents a specific state of all files and directories in the project.
Users are able to meticulously specify the files or changes in files that should be added to a commit, they can for example only select a subset of changes which happened.
By doing so one can split a large set of changes of possibly completely unrelated changes into several commits, where each commit forms a set of logically related changes.
Git is then capable of showing the exact changes between different commits, which is called a \emph{diff} and jumping between different versions of the project, which is called a \emph{checkout}.

Git is the currently most popular tool to control a project's code with a still trending tendency~\cite{article:git-popularity}.
It enables to work with multiple developers on a single code base, as it provides several different techniques, namely the history \emph{tree}, the \emph{branch} and the \emph{merge}.
The versioning history of Git is internally represented as an directed, non-cyclic, connected graph of commits or a tree.
The commits act as \emph{nodes} and the connection to their parent commits as \emph{edges}.
Every time two edges leave a single node, a new branch is created.
Git provides the feature to name branches, whereas the main branch is per default named \emph{master}.

\begin{figure}[H]
    \includegraphics[scale=0.35]{./graphs/git-history-branch}
    \centering
    \caption{A Git commit history tree.}\label{fig:git-commit-tree}
\end{figure}

As shown in figure~\ref{fig:git-commit-tree} two developer can for example create their own branch on which they can work unimpeded.
If they finished a task and want to add their work to the master branch, they can now merge their changes.
Git then tries to automatically resolve any conflicts which might have emerged from editing the same lines in a file, if that is not possible, it marks the conflicts and allows the user to manually correct them.
After this resolution a new \emph{merge commit} is created. This merge commit represents the merge of the changes of two different branches.

With this methodology it is possible to work with many people or teams on the same project, without accidentally overwriting changes of another developer, whilst maintaining a clear history of all changes in the project.

Another important feature of Git is the possibility to set up a \emph{remote}.
A remote acts as a single source of truth a developer can \emph{push} their changes to or \emph{pull} changes from other developers.
It can for example be a distinct server, which is attached to some kind of network that is accessible by the developers.
This feature allows developers to distribute to a project, as long as they have access to this network.
Git also supports several protocols such as \ac{http} or \ac{ssh} to connect to the remote and to provide a simple user management layer.


\subsection{Git User Roles}
There exist two roles in Git, namely the \emph{committer} and the \emph{author}.
Every commit in Git contains the email addresses and the names of these two people.
The author of a commit is the person which actually contributed the changes in the files.
The committer is the person, which created the git commit.
This is important to keep track of the original author of the changes.

Lets look at the case of an author contributing code to a project in an email with an attached patch file.
If a maintainer of the project now applies the patch file and commits without setting the author, the information about the original author would be lost.
Collected data indicates that in about 89\% the author and the committer are the same person.


\subsection{Internal Representation}
Git underlying storage and management solution for files is commonly described as an mini filesystem~\cite[p.~9]{book:pro-git}, thereby I will refer to this as an \ac{fs} from now on.
Git provides a collection of high level abstraction tools to work with it's underlying \ac{fs}.
In the following I will explain the most important aspects of Git's \ac{fs} structure and management.

The representation of a single file in Git is a called \emph{blob} object~\cite[p.~56]{book:pro-git}.
A blob object is a file, which has been added to a Git \ac{fs}.
It is compressed and saved in the \inlinecode{.git/objects} directory under the respective \ac{sha1} hash of the uncompressed file.
As follows there exists a blob object for every version of every file of the project.

The \ac{sha1} hashing for unique file identifier might seem unsafe at first, but the probability of a \ac{sha1} collision is really low, roughly $10^{-45}$.
Lately Google managed to force a collision in an controlled environment in 2017, but it is really unlikely to encounter a collision under normal circumstances~\cite{techreport:sha-collision}.
This characteristic of \ac{sha1} hashing will become quite important in the design of the database later on.

As mentioned in the introduction~\ref{git-introduction} Git is used to store the state of a specific directory on any underlying \ac{os} \ac{fs}.
To represent a \ac{fs} or to simply bundle multiple Git blob objects together, Git uses the tree object.

A tree object is a file, which has a \ac{sha1} hash reference to all underlying blob and tree objects as well as their names and file permissions.
To represent a subdirectory a tree simply holds a reference to another tree object.
With these tools git manages to build it's own basic representation of a file system.

\begin{minted}[linenos]{text}
    100644 blob 11d1ee77f9a23ffcb4afa860dd4b59187a9104e9  .gitignore
    040000 tree ac0f5960d9c5f662f18697029eca67fcea09a58c  expose
    100644 blob 61b5b2808cc2c8ab21bb9caa7d469e08f875277a  install.sh
    040000 tree 8aaf336db307bdcab2f082bd710b31ddb5f9ebd4  thesis
\end{minted}
\begingroup
\captionof{listing}{A tree file example\label{lst:raw-commit}.}
\endgroup

As stated before the commit is utilized to provide an exact representation of a state of the repository's files and directories.
Just as blob object, the tree and commit files are also stored in the \inlinecode{.git/objects} directory under their respective hash.

\begin{minted}[linenos]{text}
    tree      cd7d001b696db430b898b75c633686067e6f0b76
    parent    c19b969705e5eae0ccca2cde1d8a98be1a1eab4d
    author    Arne Beer <arne@twobeer.de> 1513434723 +0100
    committer Arne Beer <arne@twobeer.de> 1513434723 +0100

    Chapter 2, acronyms
\end{minted}
\begingroup
\captionof{listing}{A commit file example\label{lst:raw-commit}.}
\endgroup

As you can see in listing~\ref{lst:raw-commit}, the commit is just another kind of file utilized by Git, which contains some metadata about a repository version:

\begin{itemize}
    \item The reference to a tree object, which represents the root directory of the commit's version of the project.
    \item A reference to one or multiple parent commits, to maintain a version history.
    \item The name and email address of the author.
    \item The name and email address of the committer.
    \item The \ac{utc} timestamps with \ac{utc} offset for the commit and author date.
    \item The commit message. A message with arbitrary text from the committer.
\end{itemize}

The commit is the most important object for this thesis.
It contains crucial information such as the date of the commit as well as the email, which is needed to identify a contributor across several commits.


\subsection{More features}

Git provides many more features, which are not necessarily important for data analysis, but which might be taken into account when aggregating the data.
In the following some functionalities will be shown, which can lead to problems or irregularities in the gathered data.

\begin{description}
    \item[rebase] \hfill \\
        It is possible to \emph{rebase} branches. For instance a rebase can rewrite the commit history and change the branch point of a branch to another commit.
        This is for example a very powerful but also dangerous tool, as it implicitly changes the timestamps of the commits of the rebased branch.

    \item[force push] \hfill \\
        Git allows to push to a remote with the \inlinecode{--force} flag, which is called a \emph{force push}.
        This enables the users to rewrite every commit in the whole history tree. If another user has a git repository version

\end{description}






\section{Data source}\label{data-source}
The acquisition and clean-up of data was the biggest initial task of this thesis.
Selecting a data source was a crucial step, as good data for analysis and evaluation is the backbone of this thesis.
This section will list all requirements in detail and evaluate why I chose to use Github as a data source.
Furthermore, some functionalities of Github will be explained and a brief overview of the data provided by Github's \ac{api} will be given.


\subsection{Requirements}\label{requirements}
The data source had to satisfy as many requirements as possible, as specified in Section~\ref{attack-goals}.

To accomplish a meaningful analysis one needs a sufficient amount of commits.
For instance, it is necessary to have a few commits per weekday over a timespan of at least a month for a simple sleep rhythm analysis.
If there are only 20 commits for a user over the past month there is probably not enough data for a representative analysis.
To gather as many commits as possible I had to get access to as many repositories, to which the targeted users contributed to, as possible.
Thereby the data source has to provide a way to dynamically explore repositories around a single user or company.

For analysis of companioned persons as described in Section~\ref{attack:industrial-spy} it is crucial to find users, who are likely to know each other.
Optimally the data source provides a functionality for users to actively mark other users as their friends or colleagues.

To attack a company or to spy on company members, as described in Section~\ref{attack:employer-monitoring}, the best case scenario would be to have full access to all repositories owned by the company.
The data source thereby needs to provide some kind of representation for a company.
Ideally, there should also be a list of all company members for evaluation purposes of data mining findings.


\begin{itemlist}{A summary of the requirements to the data source:}
    \item Real world data
    \item Large amount of repositories
    \item Access to all commits of each repository
    \item Access complete metadata for each commit
    \item Email address to user association
    \item Methods to discover repositories a user contributed to
    \item Methods to discover possibly companioned contributor
    \item A representation of a company
    \item Access to members of a company
\end{itemlist}


\subsection{Github}\label{github}
I decided to use Github as a data source, as it is not only convenient to find \acp{url} for cloning repositories but also provides solutions for most of the other requirements.
It hosts one of the biggest collections of open-source projects~\cite{techreport:how-github-conquered} with 64 million repositories, 24 million users and 1,5 million organizations~\cite{article:github-statistics}.
Github also provides a well-documented \ac{api} for querying its metadata and there are libraries for most major languages, which provide an abstraction layer for this \ac{api}.
This \ac{api} is publicly available and can be used by anyone registered on Github.

For instance, Gitlab, one of Github's competitors, has much fewer data to offer.
Gitlab does not provide detailed usage statistics, but they state that they only host about 100000 organizations, which is remarkably less than Github~\cite{article:gitlab-help}.

On the other hand, one of the downsides of using Github is, that we do not have access to all metadata.
For example, the full list of members for organizations is often inaccessible, as users need to actively opt-in to be publicly displayed as a member of the organization.
The internal team structures of organizations are not visible at all, as one needs to be a member of the organization to access those.
Another problem are dangling email addresses, which are not related to any account anymore.
All commits made with such an email address cannot be used any more for any analyses that require a user to commit relationship.
But even though some ground truth is missing, I decided to use this approach as it still was the most promising way to gather as much data and real-world noise as possible, compared to other open-source hosting services.

\begin{figure}[H]
\includegraphics[scale=0.27]{./graphs/github-data-structure}
\centering
\caption{A simplified visualization of Github's internal relationships between the most important objects in Crow's foot notation.}\label{fig:github-relationship}
\end{figure}

\subsection{Github's Features}\label{github-features}
In the following I will explain some of the features provided by Github, that cover the requirements listed in Section~\ref{requirements}.
Github offers some features, which, for example, are convenient to finding repositories a specific user contributed to or to find contributors hwo likely personally know each other.

\begin{description}
    \item[Stars] \hfill \\
        A very crucial feature is \emph{starring}. Every user has the possibility to star a repository to show appreciation or interest in this specific project.
        Hence popular repositories usually have a comparatively large number of stars. For instance, the Github Linux kernel mirror has a star count of over 58000~\footnote{`Linux kernel source tree' Github.com, https://github.com/torvalds/linux (accessed, 24.04.2018)}.
        Even though Github allows to query all repositories, which are owned or forked by a user, their \ac{api} does not provide a method to get all repositories a user ever contributed to.
        However, Github provides an endpoint to query all starred repositories of a user.
        In case a user stars a repository he contributed to, whilst not owning it, it is now possible to get this repository with this feature.
        Of course, it is still not a reliable way to get all repositories a user ever contributed to, but it is a viable approach to get at least a few of them.

    \item[Follower] \hfill \\
        Another important feature is \emph{following}.
        Every user can follow any other user to get informed, when they do specific things, like creating new repositories or starring repositories or to simply show interest in or respect for their work.
        By getting all followed or following users, one might catch some friends of the user.
        It is also possible that a user follows the owner of a repository he contributed to.
        By using this feature it is thereby possible to get some additional repositories they contributed to, as well as some friends of the user.

    \item[Organizations] \hfill \\
        The last feature is \emph{organizations}.
        An organization is used to host projects under an account, that is not necessarily led by a single natural person, but rather supports roles with different permissions and team structures.
        Github allows querying all repositories of an organization via their \ac{api}.
        This enables us to link an organization to its owned repositories and as a result to perform analyses for users on a specific organization repository subset.

        Generally, organizations provide us with some important ground truth, even though the information might not be complete.
        Despite not knowing all members of an organization, we still get some useful information to estimate the tendency of precision of our knowledge extraction algorithms.
\end{description}


\section{Data Aggregation}\label{aggregator}
As mentioned in Section~\ref{github}, I decided to use Github as my primary data source and to utilize their \emph{Github APIv3} for this purpose.
The aggregator and analysis program written for this thesis is named \emph{Gitalizer}.
In this section I will explain the technologies and methods used in the data aggregation process, the database structure and the interaction with Github's \ac{api}.
In the end some problems which occurred during the data collection will be shown as well.


\subsection{Existing Solutions}
There are a number of existing solutions for accessing and collecting git metadata.
In the following the practicability of these solutions will be evaluated based on our requirements.

\subsubsection{GHTorrent}
The \emph{GHTorrent} project aims to provide Github's metadata to elude the limitations Github's rate limiting~\cite{inproceedings:ghtorrent}.
It provides representations for followers and commits, as well as organizations and organization members, but there are some crucial informations missing.
GHTorrent only stores the main email address of a user and does thereby not support the handling of multiple emails, as commits are directly assigned to their respective Github user id.
Commits miss information about additions and deletions in lines of code, which implicates that each commit would need to be scanned by a separate program again.
Furthermore GHTorrent does not have the concept of \emph{starring}, which makes it hard to reduce the amount of repositories to scan to a manageable size.
Their database provides about 4~\ac{TB} of data according to their website, which is too much information without very precise limitation~\footnote{`The GHTorrent project' ghtorrent.com, http://ghtorrent.org/ (accessed, 05.05.2018)}
It provides a vast amount of data, but at the same time it cannot be ensure, that the data is as complete as possible for a specific user.
Modifying the GHTorrent code base and extending their database schema has thereby been judged as impractical.

\subsubsection{ghcrawler}
Microsoft provides an open-source crawler called \emph{ghcrawler}, which is supposed to continuously fetch data from Github~\footnote{`Github crawler' github.com, http://github.com/ (accessed, 05.05.2018)}.
Sadly their documentation is very bad and after diving into their source code, it appears that their crawler is for Github entities only and not for the underlying data of git repos.

\subsubsection{Alitheia-Core}
\emph{Alitheia-Core} is a Java data collector for git repositories.
It is not actively maintained since more than three years and their documentation website is offline.
Using this library seemed unpromising and unviable.

https://github.com/mauricioaniche/repodriller
\subsubsection{RepoDriller}
The \emph{RepoDriller} project is project that aims to support researchers by providing easy access to repository data from Github~\footnote{`A tool to support researchers on mining software repositories studies' github.com, http://github.com/ (accessed, 05.05.2018)}.
Despite providing a good solution for getting all necessary information from a repository, it provides no way to explore Github using \emph{stars} or \emph{following}.
These features, as well as assignment of emails to a contributor via the Github \ac{api}, would have to be added.
As the program is also written in Java and I am no longer familiar with the language and its ecosystem, I decided to stick to writing my own solution.

\subsection{Database}\label{gitalizer-database}
To store the gathered Information I chose a \ac{sql} based solution.
PostgreSQL provides excellent tools to ensure a high consistency in your database, namely check constraints, as well as great support for working with times, time zones and locations.
The \ac{sql} database is used in combination with the \ac{orm} library \emph{SQLAlchemy}.

The basic schema created for the purpose of this thesis consists of five \ac{orm} models.
A model for commits, emails, repositories, contributors and organizations has been created.
The latter is only important to validate results and is not actually used for knowledge discovery, as this is Github specific data.

\begin{figure}[H]
\includegraphics[scale=0.3]{./graphs/gitalizer-data-structure}
\centering
\caption{Gitalizer database relationships.}\label{fig:gitalizer-relationship}
\end{figure}

Every commit of each repository is saved in the database along with its \ac{sha1} hash and the two email addresses as in Listing~\ref{lst:raw-commit}.
It is important to note that there is a many-to-many relationship in figure~\ref{fig:gitalizer-relationship} between commits and repositories.
This feature prevents duplication of the same commits from forked repositories.
It is, for instance, a common practice to create a fork of a repository to develop without intervening with the main git repository of the project.
As described in Section~\ref{git-internal-representation} the probability of a \ac{sha1} collision is highly improbable.
By exploiting this feature, it is possible to enforce a unique constraint on the commit hash column, assuming that any duplicated commit hash actually results from a forked or copied repository.
The formula for calculating the probability of such a collision is:

\begin{equation}\label{eq:collision-probability}
    p \leq \frac{n(n-1)}{2} * \frac{1}{2^{b}}
\end{equation}

Without this mechanism it could be possible that the same commit of a contributor could be used multiple times as a result of repository forking.
After collecting 43 million commits from Github and actively ignoring obvious project forks, there are still 49 million references between commits and repositories.
This means that about 13\% of gathered commits result from forked repositories which can not easily be identified as such.
Considering Formula~\ref{eq:collision-probability}, the probability of a collision for 49 million hashes on the 160 bit \ac{sha1} hash would be about $8.21 * 10^{-34}$.

As stated above each commit is also saved with its respective email addresses.
There exists a one-to-many relationship between contributors and emails, as every contributor can obtain an unlimited amount of email addresses.
It is necessary to connect all email addresses commit to a specific contributor, to unambiguously assign all commit to their respective contributor.
This relationship does not have a \inlinecode{NOT NULL} constraint as it happens quite often that an email address can not be assigned to any person.
Looking at the collected data it appears that roughly 20\% of all collected email addresses from Github are no longer connected to an active user.

As stated in Section~\ref{github-features} Github provides a way to organize several people in organizations and teams.
As one of the goals of this thesis is to see if it is possible to detect member of an organization in open-source projects, a model for organization has been created as well.
This data can then be used to check against the results of this research's results.


\subsection{Gitalizer}
The Program written for this thesis features data aggregation, preprocessing, knowledge extraction and visualization.
Gitalizer uses a PostgreSQL database for data storage and data consistency checks as described in~\ref{gitalizer-database}.
For interaction with the Github \ac{api} the \emph{PyGithub} library is used, which provides a convenient abstraction layer for requests and automatically maps \ac{json} responses to \emph{Python} objects.

The data aggregation module of Gitalizer is capable of several approaches for gathering data.
In the following we will look at those approaches in detail.

\begin{description}
    \item[Git repository]\label{stand-alone-repository-scan} \hfill \\
        Gitalizer can scan any git repository from a \ac{ssh} or \ac{http} \acs{url} as long as the current user has access to it.
        The repository is cloned into a local directory. After the cloning finished the actual scanning process begins.
        During the scan, we git checkout the HEAD of the current default branch for this repository and walk down every reachable commit of the Git history.
        The program saves all available metadata for each commit in its database, namely the emails, timestamps as well as additions and deletions to the project in lines of code.

        After this scan we are still missing a lot of information.
        There exists no unique identifier of an author or committer, as names may change or can be ambiguous and a single person can have multiple email addresses.
        The problem with the simplicity of Git is that there exists no concept of an user.
        Thereby we cannot easily link several email addresses to a specific contributor without additional metadata.


    \item[Github Repository]\label{github-repo-scan} \hfill \\
        To tackle the problem of missing metadata in~\ref{stand-alone-repository-scan}, I used the Github \ac{api} to get some of the missing metadata.
        The general approach is the same as in the previous scan method. The repository is cloned and locally scanned.
        However, a request to Github is issued every time an email is found, which we do not already have linked to a contributor.
        Github allows to link multiple email addresses with a single user account and automatically references the respective user in their own \ac{api} commit representation.
        With this additional metadata we gain ground truth about the identity of an author or committer.

        Anyway this approach does not work, if the user of a commit removes the email, which has been used for the commit, from his account or if the user completely deletes their account.
        If this happens and the email contributor relationship has not already been created, there is nothing that can be done and these commits need to be handled later on in the preprocessing of the data.

    \item[Github User]\label{github-repo-scan} \hfill \\
        To try getting all repositories of a specific user, a new functionality has been added, which highly utilizes the Github \ac{api}.
        At first several requests are issued to get all repositories of the specified user, as well as all starred repositories of this user.
        During the repository exploration, every relevant repository is added to a shared queue, lets call it ``repo-queue'', which is then processed by a multiprocessing pool of workers.
        Each worker process pops another entry from the ``repo-queue'' and scans a single repository as described in~\ref{github-repo-scan}.


    \item[Connected users and organizations]\label{github-repo-scan} \hfill \\
        For detection and analysis of connections between contributors over multiple repositories additional user repository discovery as described in~\ref{requirements}, another feature has been added to Gitalizer.
        Gitalizer is able to achieve this by not just scanning a single user, but also scanning the repositories of all following and followed users as described in~\ref{github-user-scan}.
        For this task two different worker pools are utilized.
        The user discovery pool is initialized with a shared queue, lets call it ``user-queue'', of all users we need to look at.
        This worker pool simply searches for relevant repositories of each user and passes the repository \ac{url} to a second shared queue.
        The second pool then processes this ``repo-queue'' as described in~\ref{github-repo-scan}.

        For organizations it is nearly the same approach.
        Initially all repositories, which are owned by the organization, are added to the ``repo-queue''.
        All publicly visible organization members are then added to the ``user-queue'' and processed as described above.
\end{description}


\subsection{Database Optimization}
As the database kept growing, it became the bottleneck in the aggregation process several times.
As a result, adjustments in the database schema, PostgreSQL parameter tweaking and migration to better hardware were necessary.
The first considerable slowdown occurred after reaching about 12~\acp{gb} of data.
At this stage the database write and read operations took longer than the actual aggregation process and the whole machine started to become unresponsive because of high I/O load.

The performance of the database then needed to be continuously tweaked in several steps.
The first countermeasure was the reduction of commits using deduplication by using the features of the \ac{sha1} hash as stated in Section~\ref{database-design}
The ignoring of forked repositories reduced the amount of entries in the relation table between commits and repositories by another 26\%.

The next performance boosts were achieved by disabling or loosening several fail-safe mechanisms of PostgreSQL, namely `synchronous commit' and `write ahead' parameter, which are designed to save data on a system crash.
As there is no important or critical data handled it was acceptable to pass on these mechanisms, and trade safety for performance.

\begin{figure}[H]
\includegraphics[scale=0.22]{./graphs/server-graphs/query-refactoring}
\centering
\caption{The CPU load of the aggregation server during optimization.}\label{fig:cpu-load}
\end{figure}

After renting a root server and deploying Gitalizer onto it, the aggregation process worked really well, until the database size hit about 25~\ac{gb}.
In Figure~\label{fig:cpu-load} you see the overall \ac{cpu} load right before optimizing several \ac{sql} queries by caching intermediate results and increasing the cache size of PostgreSQL.
The dark blue represents the I/O wait time while the light blue represents the actually used processor capacity by the aggregator.
Due to complex and numerous \ac{sql} queries the server became partly unresponsive and the aggregation process began to stall.

After many improvements the server can now run with 16 worker threads and roughly 38~\ac{gb} of data without any signs of slowdown whilst using the rate limit to capacity.


\subsection{Incremental Aggregation}
To ensure a constantly up to date database, Gitalizer needed to be capable of fast rescans of repositories.
The initial scan of a repository always includes cloning, scanning the whole repository and writing it into the database.
After a repository is scanned completely at least once it is marked as as such and will never by completely scanned again.
All following runs then only clone the repository and scan the newest unknown commits.
These are discovered by performing a breadth first search until no new nodes are found.

\begin{figure}[H]
\includegraphics[scale=0.3]{./graphs/git-history-rewrite}
\centering
\caption{Gitalizer database relationships.}\label{fig:gitalizer-relationship}
\end{figure}

As explained in Section~\ref{more-git-features} it is possible to rewrite commits and force push them.
This scenario needs to be explicitly handled since force pushes can completely alter the history of a git repository, which can subsequently lead to a split in the Git history and leaves dangling commits.
As the complete history of a repository is stored inside the database, Gitalizer needs to detect a force push by walking down the git history tree until it finds known commits.
If any of these commits has children, which are not in the newly scanned commits, a force push took place and the old commit history has to be truncated, since it is now outdated and irrelevant.
An example scenario can be seen in Figure~\ref{fig:gitalizer-relationship}, where all red commits represent the old commit history, which needs to be truncated.


\subsection{Problems}
During the development of the data aggregator I experienced a few problems and edge cases which needed to be handled.
The earliest and most delaying problem was the rate limit of the Github \ac{api}, which limits to 5000 requests per hour.
Beside this rate limiting there also is an abuse detection mechanism, which triggers, if too many requests are fired in a short amount of time.
The solution for this problem resulted in various hacks, which include random wait times to detain those mechanisms from triggering.

The first version of the aggregator did not clone and scan the repository locally, but rather gathered all information from the Github \ac{api} endpoints.
This approach worked well until the aggregator hit the official repository of \emph{Nmap}, which has about 11.000 commits and took over three hours to scan.
Soon I realized that this would severely slow down my research and I then started to continuously minimize the amount \ac{api} calls issued by Gitalizer.
A connected user scan of my own Github account led to about 600.000 commits and took about one and a half day on the final working version of Gitalizer, to provide you with a reference of scale.

After implementing multiprocessing, I managed to hit the rate limit again, as I was now issuing requests to the \ac{api} with multiple threads.
To fix this issue I implemented a wait and retry wrapper around every single function call or object access, which triggered a call to the Github \ac{api}.
Afterwards the aggregator was capable of running multiple days without worker processes silently dying or finishing with incomplete data.

Fine tuning the edge cases and the handling of the \ac{api} took about 3 months, since there were many problems such as unpredictable error responses from Github, missing data in queries or simply unknown or broken encodings in Github's metadata.

A big throwback became apparent as I discovered that PostgreSQL automatically normalizes \ac{utc} timestamps with any offset to the \inlinecode{UTC +0} timezone.
As a result of this normalization, the exact time of the commit admittedly stays the same, but the crucial metadata about the offset is lost.
As a consequence the commit model needed to be adjusted, as the \ac{utc} offset had to be stored explicitly, and the whole commit aggregation process was started from scratch.

Another problem occurred during the local scanning of the repositories.
The library used for interaction with Git \emph{libgit2} issued \emph{stat} Linux syscalls during a diff operation for each file, which changed between those commits, to check if there were any local uncommitted changes.
Anyhow the repositories, which were locally scanned, were cloned with \emph{bare} mode.
This means that there exists no project root directory, but rather only the git internal representations of those files, which makes the behaviour stated above unnecessary and unwanted.
As a result all processes slowed severely down due to high I/O wait times, because of stat syscalls on non existent files.
Luckily after reporting the issue~\footnote{`Unnecessary syscalls on bare repository' github.com, https://github.com/libgit2/libgit2/issues/4480 (accessed, 25.04.2018)} it was resolved in a week and I was able to continue developing with my own compiled version of the libgit2 library.



\chapter{Implementation}\label{implementation}
After collecting all necessary data as shown in Chapter~\ref{data}, I will now begin to analyse this data.
In this chapter the approach for several attacks, as listed in Section~\ref{attack-models} will be introduced and the attack's goals recapitulated.
The possible applications for the gained knowledge will be stated and the implementation of and requirements to the respective algorithm will be explained for each attack.

\section{Holiday and Sick Leave Detection}

The information about anomalies in the regular work pattern can be a valuable information for several parties.
Usually only few parties people know about the holiday or sick leave times of a person.
To know if a persons tends to become sick often or for long times is a dangerous intrusion into a persons privacy.
For instance this could be abused by head hunters or personnel managers to cull possible employees with too high sick leave rates and thereby reduce the job prospects of the target.

For employees this might convenient to detect anomalies in the productivity of an employee.
In case an employee doesn't commit on a regular basis for several days, this behaviour would be instantly visible with this method.

Another attack vector could be to look at the correlation of miss-out between several employees.
This attack could even be performed by an outsider, if the employees of a company are known.
The information gained by this attack could be quite delicate, as they could reveal relationships between employees.
This attack is heavily inspired by an article about data mining articles from the popular German weekly magazine \emph{Der Spiegel} written by the David Kriesel~\cite{article:spiegel-mining}.


\subsection{Implementation}

The requirements for this algorithm is the detection of a regular work pattern for a given interval.
It must have the ability to adjust to a changing work pattern, but at the same time has to be capable of detecting anomalies in this pattern.
If the attacker wants to look at multiple people, some kind of measure for similarity in the missing time patterns has to exist.

The input for this analysis is the intersection between all commits from the considered repositories and all commits from the considered contributors.
The commits' meta data used for this analysis are time stamps as well as additions and deletions in lines of code.

\begin{figure}[H]
    \includegraphics[scale=0.20]{./graphs/analysis/work-time-analysis}
    \centering
    \caption{The work time analysis of the author.}\label{fig:missing-time}
\end{figure}

The analysis of the data is a chronological scan of all commits for specific user.
Before performing the actual analysis, the data is converted into a usable format representing the week days.
The converted data format represents the amount of commits for each day in the last year.
It is really difficult to measure productivity in lines of code committed or in the amount of commits made by a person, as they don not necessarily display the amount of work that have been put into those commits.
As a result I decided, that a day counts as a work day as long as at least single commit has been made during the day.

\begin{minted}{python}
def analyse(weeks):
    prototype = None
    for index, week in weeks.items():
        next_six_weeks = weeks[index:index+future_lookup]
        if not prototype:
            # See if there is a prototype in the next few weeks.
            prototype = find_prototype(next_six_weeks)

            # Check if this specific week is a anomaly
            check_anomaly(prototype, week)

            continue

        prototype_exists = prototype_exists_in_next_weeks(next_six_weeks)
        if not prototype_exists:
            # We couldn't find the prototype in the next few rows
            # Try to find a new prototype
            prototype = find_prototype(next_six_weeks)

        check_anomaly(prototype, week)


def check_anomaly(prototype, week):
    if week.working_days == 0:
        save_anomaly(week)

    if prototype is not None:
        different_days = week.working_days - prototype.working_days
        // A single day variance is acceptable
        if different_days >= 1:
            save_anomaly(week)

\end{minted}
\begingroup
\captionof{listing}{Miss-out analysis \label{lst:naive-path-tracing}.}
\endgroup

The algorithm inspects every week work pattern of a given interval.
At the beginning a new \emph{prototype} is tried to be found.
A prototype is a representative week work pattern, which resembles the average work day pattern of the next weeks.
This happens in the function \inlinecode{find\_prototype}.
It performs a simple iteration over a given interval to find a work day pattern, which occurs more often than a given threshold.
If a prototype is found, we are capable of identifying anomalies that deviate from this pattern.

For each following week it is firstly checked if this week is a anomaly for this prototype.
Anomalies are simply detected by comparing the amount of working days of the prototype and the currently looked at week.
The real difference in the working pattern is not suitable for this analysis, as it produces too many false positives for employees with flexible work time.

Secondly it is checked if there exists a week in the near future, which is identical to the prototype.
If there is no week identical to the prototype in the near future, the current prototype is reset and a new prototype needs to be found.

In case no prototype can be found, anomalies cannot be easily identified, as there exists no pattern to check against.
Only obvious anomalies, namely weeks without a single work day, will then be marked as such.


\subsection{Interpretation and evaluation}

Figure~\ref{fig:missing-time} shows the analysis of the author for his work repositories.
The y-axis shows the additions or deletions per commit, the x-axis shows the week of a year.
For better analysis and evaluation of the results, a scatter plot with the additions and deletions per commit has been added on top of the miss-out graph.

The evaluation of this algorithm turned out to be quite difficult, as there is no publicly available information about sick leave or holiday.
For the purpose of this thesis I had to use anonymous statistics of several friends and colleagues to evaluate the algorithm.
The algorithm successfully manages to find all anomalies, which occurred in the last year, for all seven regarded users.

An unexpected side effect of detecting prototypes is that the algorithm also also finds inconsistencies in the work routine.
For instance between week 37 to 45 in Figure~\ref{fig:missing-time} I was forced to reduce my working hours due to legal questions and shift hours and working days.
It is hard to interpret those inconsistencies without more contextual information, but nevertheless it provides the fact that something happened during this time.

\begin{figure}[H]
    \includegraphics[scale=0.20]{./graphs/analysis/work-time-analysis-comparison}
    \centering
    \caption{The miss-out analysis of several employees.}\label{fig:miss-out-comparison}
\end{figure}

In Figure~\ref{fig:miss-out-comparison} the comparison between multiple employees can be seen.
Contributor0 and Contributor2 are working on flexible work time, while the other two contributors have regular working hours, which reflects in the inconsistencies of those contributors.


\section{Working hours}

This attack aims to gather as much information about the working hour behaviour as possible.
Information which can be extracted by this attack is for example the sleep rhythm of the target.
It can be used to detect whether the target is a person working regular shifts from from Monday to Friday or rather a freelancer or open-source contributor working at the weekend.
The attack can be further used to compare the working hour patterns of several people in the same project or organization.
For instance this could be used to infer relationships between colleagues, based on an equal working shift.

\subsection{Implementation}

\begin{figure}[H]
    \includegraphics[scale=0.20]{./graphs/analysis/ordered-punchcard}
    \centering
    \caption{Punchcard of the sleep rhythm analysis of the author.}\label{fig:working-hour-rhythm-author}
\end{figure}

\begin{figure}[H]
    \includegraphics[scale=0.20]{./graphs/analysis/random-punchcard}
    \centering
    \caption{Punchcard of an user without a regular sleep rhythm.}\label{fig:random-sleep-rhythm}
\end{figure}

\begin{figure}[H]
    \includegraphics[scale=0.20]{./graphs/analysis-affinity/3}
    \centering
    \caption{The work time analysis of the author.}\label{fig:missing-time}
\end{figure}


\section{Geolocation}

First of all, it needs to be clarified, that parts of this attack only works under specific circumstances.
Git commit timestamps identical to the current local time of the underlying \ac{os}.
If one wants to show the travel path of a target, the target's \ac{os} needs to automatically adjust the \ac{utc} accordingly to the current geolocation of the device.

This feature is available for the newer versions of big \acp{os}, such as Windows~\footnote{Ivan Jenic, `Your Time Zone Can Now Switch Automatically in Windows 10', windowsreport.com, https://windowsreport.com/time-zone-automatic-switch-windows-10 (accessed, 24.04.2018)}
and Mac, but it is not necessarily enabled by default.
It is also available for Linux, for instance with the \emph{tzupdate} package~\footnote{`Set the system timezone based on IP geolocation', github.com, https://github.com/cdown/tzupdate (accessed, 24.04.2018)}, but it needs to be installed and activated manually.

\begin{figure}[H]
    \includegraphics[scale=0.10]{./graphs/analysis/author-home-location}
    \centering
    \caption{Home location analysis of the author.}\label{fig:author-home-location}
\end{figure}

In Figure~\ref{fig:author-home-location} the visualized home location analysis of the author can be seen.
Regions marked in dark green are regions, in which the contributor is likely to live.
The light green region represents the timezone of the home location.
As you can see in Figure~\ref{fig:author-home-location} the country French Guiana is also marked as a possible home location.
This problem occurs due to the several conversions between country names and codes, which were necessary as stated~\ref{timezone-implementation}.
This misassignment only happens during the visualization of the results and thereby doesn't affect the results of the analysis.

To evaluate the overall precision of the geolocation results, the correctness of the determined home location is checked.
Github allows users to specify a string for their current home, which is collected during the aggregation process.
Unfortunately there are no conventions on how this string has to look like.
Initially I tried to pass these strings to OpenStreetMap, but this resulted in too many wrongly assigned locations.
The data provided by the users was obviously too arbitrary and full of mistakes for the OpenStreetMap \ac{api} to handle.

As a result I decided to manually choose a subset of locations by looking for distinct identifiers in the location strings.
For instance, every home location of contributor, which contained \emph{Germany} or \emph{Deutschland} in their location string, should be in the \ac{utc} +1 timezone and contain the specific timezone \inlinecode{Europe/Berlin}.
I created INSERTHERE such rules and was thus able to validate the home location of about NUMBERHERE contributors.
The assignment of the contributors home location was correct in about 92\% of the considered contributors.




\chapter{Evaluation and Interpretation}\label{evaluation}
In the last chapter I showed the implementation of several possible attacks, which could be performed on the gathered data.
This Chapter will now attend to the evaluation of all results gained from these attacks.
I will present the extracted information from each algorithm and compare it to real world ground truth.
This information will be then be explained and audited in terms of precision and reliability.

\section{Holiday and Sick Leave Detection}

The information about anomalies in the regular work pattern can be a valuable information for several parties.
Usually only few parties people know about the holiday or sick leave times of a person.
To know if a persons tends to become sick often or for long times is a dangerous intrusion into a persons privacy.
For instance this could be abused by head hunters or personnel managers to cull possible employees with too high sick leave rates and thereby reduce the job prospects of the target.

For employees this might convenient to detect anomalies in the productivity of an employee.
In case an employee doesn't commit on a regular basis for several days, this behaviour would be instantly visible with this method.

Another attack vector could be to look at the correlation of miss-out between several employees.
This attack could even be performed by an outsider, if the employees of a company are known.
The information gained by this attack could be quite delicate, as they could reveal relationships between employees.
This attack is heavily inspired by an article about data mining articles from the popular German weekly magazine \emph{Der Spiegel} written by the David Kriesel~\cite{article:spiegel-mining}.


\subsection{Implementation}

The requirements for this algorithm is the detection of a regular work pattern for a given interval.
It must have the ability to adjust to a changing work pattern, but at the same time has to be capable of detecting anomalies in this pattern.
If the attacker wants to look at multiple people, some kind of measure for similarity in the missing time patterns has to exist.

The input for this analysis is the intersection between all commits from the considered repositories and all commits from the considered contributors.
The commits' meta data used for this analysis are time stamps as well as additions and deletions in lines of code.

\begin{figure}[H]
    \includegraphics[scale=0.20]{./graphs/analysis/work-time-analysis}
    \centering
    \caption{The work time analysis of the author.}\label{fig:missing-time}
\end{figure}

The analysis of the data is a chronological scan of all commits for specific user.
Before performing the actual analysis, the data is converted into a usable format representing the week days.
The converted data format represents the amount of commits for each day in the last year.
It is really difficult to measure productivity in lines of code committed or in the amount of commits made by a person, as they don not necessarily display the amount of work that have been put into those commits.
As a result I decided, that a day counts as a work day as long as at least single commit has been made during the day.

\begin{minted}{python}
def analyse(weeks):
    prototype = None
    for index, week in weeks.items():
        next_six_weeks = weeks[index:index+future_lookup]
        if not prototype:
            # See if there is a prototype in the next few weeks.
            prototype = find_prototype(next_six_weeks)

            # Check if this specific week is a anomaly
            check_anomaly(prototype, week)

            continue

        prototype_exists = prototype_exists_in_next_weeks(next_six_weeks)
        if not prototype_exists:
            # We couldn't find the prototype in the next few rows
            # Try to find a new prototype
            prototype = find_prototype(next_six_weeks)

        check_anomaly(prototype, week)


def check_anomaly(prototype, week):
    if week.working_days == 0:
        save_anomaly(week)

    if prototype is not None:
        different_days = week.working_days - prototype.working_days
        // A single day variance is acceptable
        if different_days >= 1:
            save_anomaly(week)

\end{minted}
\begingroup
\captionof{listing}{Miss-out analysis \label{lst:naive-path-tracing}.}
\endgroup

The algorithm inspects every week work pattern of a given interval.
At the beginning a new \emph{prototype} is tried to be found.
A prototype is a representative week work pattern, which resembles the average work day pattern of the next weeks.
This happens in the function \inlinecode{find\_prototype}.
It performs a simple iteration over a given interval to find a work day pattern, which occurs more often than a given threshold.
If a prototype is found, we are capable of identifying anomalies that deviate from this pattern.

For each following week it is firstly checked if this week is a anomaly for this prototype.
Anomalies are simply detected by comparing the amount of working days of the prototype and the currently looked at week.
The real difference in the working pattern is not suitable for this analysis, as it produces too many false positives for employees with flexible work time.

Secondly it is checked if there exists a week in the near future, which is identical to the prototype.
If there is no week identical to the prototype in the near future, the current prototype is reset and a new prototype needs to be found.

In case no prototype can be found, anomalies cannot be easily identified, as there exists no pattern to check against.
Only obvious anomalies, namely weeks without a single work day, will then be marked as such.


\subsection{Interpretation and evaluation}

Figure~\ref{fig:missing-time} shows the analysis of the author for his work repositories.
The y-axis shows the additions or deletions per commit, the x-axis shows the week of a year.
For better analysis and evaluation of the results, a scatter plot with the additions and deletions per commit has been added on top of the miss-out graph.

The evaluation of this algorithm turned out to be quite difficult, as there is no publicly available information about sick leave or holiday.
For the purpose of this thesis I had to use anonymous statistics of several friends and colleagues to evaluate the algorithm.
The algorithm successfully manages to find all anomalies, which occurred in the last year, for all seven regarded users.

An unexpected side effect of detecting prototypes is that the algorithm also also finds inconsistencies in the work routine.
For instance between week 37 to 45 in Figure~\ref{fig:missing-time} I was forced to reduce my working hours due to legal questions and shift hours and working days.
It is hard to interpret those inconsistencies without more contextual information, but nevertheless it provides the fact that something happened during this time.

\begin{figure}[H]
    \includegraphics[scale=0.20]{./graphs/analysis/work-time-analysis-comparison}
    \centering
    \caption{The miss-out analysis of several employees.}\label{fig:miss-out-comparison}
\end{figure}

In Figure~\ref{fig:miss-out-comparison} the comparison between multiple employees can be seen.
Contributor0 and Contributor2 are working on flexible work time, while the other two contributors have regular working hours, which reflects in the inconsistencies of those contributors.


\section{Working hours}

This attack aims to gather as much information about the working hour behaviour as possible.
Information which can be extracted by this attack is for example the sleep rhythm of the target.
It can be used to detect whether the target is a person working regular shifts from from Monday to Friday or rather a freelancer or open-source contributor working at the weekend.
The attack can be further used to compare the working hour patterns of several people in the same project or organization.
For instance this could be used to infer relationships between colleagues, based on an equal working shift.

\subsection{Implementation}

\begin{figure}[H]
    \includegraphics[scale=0.20]{./graphs/analysis/ordered-punchcard}
    \centering
    \caption{Punchcard of the sleep rhythm analysis of the author.}\label{fig:working-hour-rhythm-author}
\end{figure}

\begin{figure}[H]
    \includegraphics[scale=0.20]{./graphs/analysis/random-punchcard}
    \centering
    \caption{Punchcard of an user without a regular sleep rhythm.}\label{fig:random-sleep-rhythm}
\end{figure}

\begin{figure}[H]
    \includegraphics[scale=0.20]{./graphs/analysis-affinity/3}
    \centering
    \caption{The work time analysis of the author.}\label{fig:missing-time}
\end{figure}


\section{Geolocation}

First of all, it needs to be clarified, that parts of this attack only works under specific circumstances.
Git commit timestamps identical to the current local time of the underlying \ac{os}.
If one wants to show the travel path of a target, the target's \ac{os} needs to automatically adjust the \ac{utc} accordingly to the current geolocation of the device.

This feature is available for the newer versions of big \acp{os}, such as Windows~\footnote{Ivan Jenic, `Your Time Zone Can Now Switch Automatically in Windows 10', windowsreport.com, https://windowsreport.com/time-zone-automatic-switch-windows-10 (accessed, 24.04.2018)}
and Mac, but it is not necessarily enabled by default.
It is also available for Linux, for instance with the \emph{tzupdate} package~\footnote{`Set the system timezone based on IP geolocation', github.com, https://github.com/cdown/tzupdate (accessed, 24.04.2018)}, but it needs to be installed and activated manually.

\begin{figure}[H]
    \includegraphics[scale=0.10]{./graphs/analysis/author-home-location}
    \centering
    \caption{Home location analysis of the author.}\label{fig:author-home-location}
\end{figure}

In Figure~\ref{fig:author-home-location} the visualized home location analysis of the author can be seen.
Regions marked in dark green are regions, in which the contributor is likely to live.
The light green region represents the timezone of the home location.
As you can see in Figure~\ref{fig:author-home-location} the country French Guiana is also marked as a possible home location.
This problem occurs due to the several conversions between country names and codes, which were necessary as stated~\ref{timezone-implementation}.
This misassignment only happens during the visualization of the results and thereby doesn't affect the results of the analysis.

To evaluate the overall precision of the geolocation results, the correctness of the determined home location is checked.
Github allows users to specify a string for their current home, which is collected during the aggregation process.
Unfortunately there are no conventions on how this string has to look like.
Initially I tried to pass these strings to OpenStreetMap, but this resulted in too many wrongly assigned locations.
The data provided by the users was obviously too arbitrary and full of mistakes for the OpenStreetMap \ac{api} to handle.

As a result I decided to manually choose a subset of locations by looking for distinct identifiers in the location strings.
For instance, every home location of contributor, which contained \emph{Germany} or \emph{Deutschland} in their location string, should be in the \ac{utc} +1 timezone and contain the specific timezone \inlinecode{Europe/Berlin}.
I created INSERTHERE such rules and was thus able to validate the home location of about NUMBERHERE contributors.
The assignment of the contributors home location was correct in about 92\% of the considered contributors.




\chapter{Conclusion and Outlook}

The study set out to determine how feasible and precise data mining attacks on simple Git metadata could be.
All three performed attacks, lead to promising results and showed potential for malicious usage.

The miss-out analysis showed that it is possible to automatically detect holiday and sick-leave anomalies.
Additionally, it is capable of detecting other anomalies in the developer's work pattern.

Analyzing the Git commit timestamps to narrow down the geographic location of a user-led to a significant reduction of possible locations on the globe.
With a proper test group, it is also likely to prove, that the other detected timezones represent the travel history of the target.

The analysis of punch cards showed, that it is possible to detect developers working at regular five day office hours and to distinguish between working employees and leisure time developers.

However, it must be noted that in all attacks only a small amount of the available data was used.
Simply using the Git commit timestamps allowed us to perform analyses such as narrowing down the location of a contributor.
The possible applications for the remaining data, like actual changes in code, references of contributors between repositories or commit messages, are extensive.

If one would add additional data from Github, such as followers, stars or information from their issues system, the results could become even more accurate.
Developers around the world provide metadata about themselves on a daily basis, probably without knowing how much they are actually exposing.
To prevent the unauthorized usage and abuse of this data, we need to create countermeasures or prevent exposing this data in the first place.

Luckily the \ac{eu} set an example by enforcing the \ac{gdpr}, which is a regulation that strictly rules the handling of any user data.
But there are still many countries left in the world, that do not have such strict rules and that might need ways to protect their privacy from being invaded.

While \emph{Gitalizer} is a foundation for data aggregation and the conduct of rather simple analyses, there is a necessity for more detailed research with better sources for ground truth.
Additionally, more statistics about the mining process would be convenient for evaluating the research results, such as the ratio between starred and contributed repositories.
\emph{Gitalizer} is a quite complex program, but it is well documented and should allow other people to easily jump into using it.

Many of the attacks mentioned in Section~\ref{attack-goals} are not implemented, as they did not fit in the scope of this thesis.
Implementing those could be the topic of another bachelor thesis or for a subsequent master thesis.

Furthermore, it would be interesting to explore the possibilities of countermeasures such as obfuscating Git commit timestamps.



\begingroup
    \footnotesize
    \listoffigures
    \let\clearpage\relax
    \listoflistings{}
    \listoftables
\endgroup

\renewcommand*{\bibfont}{\footnotesize}
\printbibliography{}


\chapter*{Eidesstattliche Erklärung}
\onehalfspace{}
„Hiermit versichere ich an Eides statt, dass ich die vorliegende Arbeit im
Studiengang Informatik selbstständig verfasst und keine anderen als die
angegebenen Hilfsmittel – insbesondere keine im Quellenverzeichnis nicht
benannten Internet-Quellen – benutzt habe. Alle Stellen, die wörtlich oder
sinngemäß aus Veröffentlichungen entnommen wurden, sind als solche kenntlich
gemacht. Ich versichere weiterhin, dass ich die Arbeit vorher nicht in einem
anderen Prüfungsverfahren eingereicht habe und die eingereichte schriftliche
Fassung der auf dem elektronischen Speichermedium entspricht.“
\singlespace{}

\vspace{1cm}

\begin{tabular}{ll}
    \centering
    \makebox[5cm]{\hrulefill} & \makebox[5cm]{\hrulefill}\\
    Ort, Datum & Unterschrift \\
\end{tabular}


\end{document}
